\documentclass[../psets.tex]{subfiles}

\pagestyle{main}
\renewcommand{\leftmark}{Problem Set \thesection}

\begin{document}




\section{Mechanical Waves}
\begin{enumerate}[label={\arabic*)}]
    \item \marginnote{8/9:}\textcite{bib:YoungFreedman}: Problem 15.9.\par
    Which of the following wave functions satisfies the wave equation?
    \begin{enumerate}
        \item $y(x,t)=A\cos(kx+\omega t)$.
        \item $y(x,t)=A\sin(kx+\omega t)$.
        \item $y(x,t)=A(\cos kx+\cos\omega t)$.
        \item For the wave of part (b), write the equations for the transverse velocity and transverse acceleration of a particle at point $x$.
    \end{enumerate}
    \item \textcite{bib:YoungFreedman}: Problem 15.26.\par
    A fellow student with a mathematical bent tells you that the wave function of a traveling wave on a thin rope is $y(x,t)=(\SI{2.30}{\milli\meter})\cos[(\SI{6.98}{\radian\per\meter})x+(\SI{742}{\radian\per\second})t]$. Being more practical, you measure the rope to have a length of $\SI{1.35}{\meter}$ and a mass of $\SI{0.00338}{\kilo\gram}$. You are then asked to determine:
    \begin{enumerate}
        \item Amplitude.
        \item Frequency.
        \item Wavelength.
        \item Wave speed.
        \item Direction the wave is traveling.
        \item Tension in the rope.
        \item Average power transmitted by the wave.
    \end{enumerate}
    \item \textcite{bib:YoungFreedman}: Problem 15.30.\par
    \textbf{Interference of Triangular Pulses.} Two triangular wave pulses are traveling toward each other on a stretched string as shown below. Each pulse is identical to the other and travels at $\SI{2.00}{\centi\meter\per\second}$. The leading edges of the pulses are $\SI{1.00}{\centi\meter}$ apart at $t=0$. Sketch the shape of the string at $t=\SI{0.250}{\second}$, $t=\SI{0.500}{\second}$, $t=\SI{0.750}{\second}$, $t=\SI{1.000}{\second}$, and $t=\SI{1.250}{\second}$.
    \begin{center}
        \begin{tikzpicture}[scale=1.2]
            \footnotesize
            \draw [orx,very thick] (0,0)
                -- (1,0)
                -- (2,1)
                -- (3,0)
                -- (4,0)
                -- (5,1)
                -- (6,0)
                -- (7,0)
            ;
    
            \draw [very thin,|-|] (0.85,0.05) -- node[left]{$\SI{1.00}{\centi\meter}$} ++(0,0.95);
            \draw [very thin,|-|] (1,-0.15) -- node[below]{$\SI{1.00}{\centi\meter}$} ++(1,0);
            \draw [very thin,|-|] (2,-0.15) -- node[below]{$\SI{1.00}{\centi\meter}$} ++(1,0);
            \draw [very thin,|-|] (3,0.3) -- node[above]{$\SI{1.00}{\centi\meter}$} ++(1,0);
            \draw [very thin,|-|] (4,-0.15) -- node[below]{$\SI{1.00}{\centi\meter}$} ++(1,0);
            \draw [very thin,|-|] (5,-0.15) -- node[below]{$\SI{1.00}{\centi\meter}$} ++(1,0);
            \draw [very thin,|-|] (6.15,0.05) -- node[right]{$\SI{1.00}{\centi\meter}$} ++(0,0.95);
    
            \draw [grx,ultra thick,-latex] (2,1.1) -- node[above=1mm,black]{$v=\SI{2.00}{\centi\meter\per\second}$} ++(1,0);
            \draw [grx,ultra thick,-latex] (5,1.1) -- node[above=1mm,black]{$v=\SI{2.00}{\centi\meter\per\second}$} ++(-1,0);
        \end{tikzpicture}
    \end{center}
    \item \textcite{bib:YoungFreedman}: Problem 15.32.\par
    \textbf{Interference of Rectangular Pulses.} The below figure shows two rectangular wave pulses on a stretched string traveling toward each other. Each pulse is traveling with a speed of $\SI{1.00}{\milli\meter\per\second}$ and has the height and width shown in the figure. If the leading edges of the pulses are $\SI{8.00}{\milli\meter}$ apart at $t=0$, sketch the shape of the string at $t=\SI{4.00}{\second}$, $t=\SI{6.00}{\second}$, and $t=\SI{10.0}{\second}$.
    \begin{center}
        \begin{tikzpicture}[scale=1.2]
            \footnotesize
            \draw [orx!60!black,line join=round,double=orx,double distance=1.2pt] (0,0)
                -- (1,0)
                -- (1,0.75)
                -- (2,0.75)
                -- (2,0)
                -- (4,0)
                -- (4,-1)
                -- (5,-1)
                -- (5,0)
                -- (6,0)
            ;
    
            \draw [very thin,|-|] (0.5,0.05) -- node[xshift=-0.05cm,fill=white,inner sep=2pt]{$\SI{3.00}{\milli\meter}$} ++(0,0.7);
            \draw [very thin,|-|] (1,0.9) -- node[above]{$\SI{4.00}{\milli\meter}$} ++(1,0);
            \draw [very thin,|-|] (2,-0.15) -- node[below]{$\SI{8.00}{\milli\meter}$} ++(1.95,0);
            \draw [very thin,|-|] (4,-1.15) -- node[below]{$\SI{4.00}{\milli\meter}$} ++(1,0);
            \draw [very thin,|-|] (5.5,-0.05) -- node[xshift=0.05cm,fill=white,inner sep=2pt]{$\SI{4.00}{\milli\meter}$} ++(0,-0.95);
    
            \draw [grx,ultra thick,-latex] (2.1,0.4) -- ++(0.5,0) node[right,black]{$v=\SI{1.00}{\milli\meter\per\second}$};
            \draw [grx,ultra thick,-latex] (3.9,-0.75) -- ++(-0.5,0) node[left,black]{$v=\SI{1.00}{\milli\meter\per\second}$};
        \end{tikzpicture}
    \end{center}
    \item \textcite{bib:YoungFreedman}: Problem 15.34.\par
    Adjacent antinodes of a standing wave on a string are $\SI{15.0}{\centi\meter}$ apart. A particle at an antinode oscillates in simple harmonic motion with amplitude $\SI{0.850}{\centi\meter}$ and period $\SI{0.0750}{\second}$. The string lies along the $+x$-axis and is fixed at $x=0$.
    \begin{enumerate}
        \item How far apart are the adjacent nodes?
        \item What are the wavelength, amplitude, and speed of the two traveling waves that form this pattern?
        \item Find the maximum and minimum transverse speeds of a point at an antinode.
        \item What is the shortest distance along the string between a node and an antinode?
    \end{enumerate}
    \item \textcite{bib:YoungFreedman}: Problem 15.64.\par
    A strong string of mass $\SI{3.00}{\gram}$ and length $\SI{2.20}{\meter}$ is tied to supports at each end and is vibrating in its fundamental mode. The maximum transverse speed of a point at the middle of the string is $\SI{9.00}{\meter\per\second}$. The tension in the string is $\SI{330}{\newton}$.
    \begin{enumerate}
        \item What is the amplitude of the standing wave at its antinode?
        \item What is the magnitude of the maximum transverse acceleration of a point at the antinode?
    \end{enumerate}
    \item A harmonic wave travels down a string in the $+x$ direction. At position $x=0$ and time $t=0$, the following is observed: the displacement of the string is $+\SI{1.0}{\centi\meter}$, the transverse velocity is $-\SI{2.0}{\centi\meter\per\second}$, and the transverse acceleration is $-\SI{4.0}{\centi\meter\per\square\second}$.
    \begin{enumerate}
        \item What is the frequency of the wave?
        \item What is the amplitude of the wave?
    \end{enumerate}
    \item A long, uniform rope of length $L$ hangs vertically. The only tension in the rope is that produced by its own weight.
    \begin{enumerate}
        \item Show that, as a function of the distance $y$ from the lower end of the rope, the speed of a transverse wave pulse on the rope is $\sqrt{gy}$.
        \item How much time does it take for a wave pulse to travel from one end of the rope to the other?
    \end{enumerate}
    \item Using continuity conditions on a string, we derived the relative amplitudes for transmitted and reflected waves at a boundary. Show that the average power of the \emph{transmitted} wave plus the average power of the \emph{reflected} wave is equal to the average power of the \emph{incident} wave. (Otherwise, energy would not be conserved.)
\end{enumerate}




\end{document}