\documentclass[../notes.tex]{subfiles}

\pagestyle{main}
\renewcommand{\chaptermark}[1]{\markboth{\chaptername\ \thechapter\ (#1)}{}}
\setcounter{chapter}{35}

\begin{document}




\chapter{Diffraction}
\section{Single Slit Diffraction}
\begin{itemize}
    \item \marginnote{8/17:}Shining light through only one slit still yields an interference pattern.
    \item We explain this with \textbf{diffraction}.
    \item Finding the location of minima on the screen:
    \begin{figure}[h!]
        \centering
        \begin{tikzpicture}
            \footnotesize
            \draw (0,-3) -- (0,-1) (-0.1,0) -- (0.1,0) (0,1) -- (0,3);
            \draw [very thin,dashed] (0,0) coordinate (O) -- (1,0) coordinate (N);
    
            \draw [blx,very thick]
                (-1.5,-2) -- ++(0,4)
                (-2,-2) -- ++(0,4)
                (-2.5,-2) -- ++(0,4)
            ;
    
            \draw [rex,thick,-latex] (0,1) node[circle,fill,inner sep=1.5pt,label={[yshift=2pt,black]below:$P_1$}]{} -- node[above,black]{$r_1$} ++(2,1);
            \draw [rex,thick,-latex] (0,0) node[circle,fill,inner sep=1.5pt,label={[yshift=2pt,black]below:$P_2$}]{} -- node[above,black]{$r_2$} ++(2,1) coordinate (r0);
    
            \draw [very thin,|-|] (-1.1,-1) -- node[left]{$a$} ++(0,2);
            \draw [very thin,|-|] (-0.5,-1) -- node[left]{$\frac{a}{2}$} ++(0,1);
            \draw [very thin,|-|] (-0.5,0) -- node[left]{$\frac{a}{2}$} ++(0,1);
            \pic [draw,angle eccentricity=1.3,pic text={$\theta$}] {angle=N--O--r0};
        \end{tikzpicture}
        \caption{Finding diffraction minima.}
        \label{fig:diffMinima}
    \end{figure}
    \begin{itemize}
        \item Let the one slit have width $a$.
        \item Only the part of the wavefront that aligns with the slit will pass through. However, according to Huygen's principle, when the light wave reaches the slit, it will act like infinitely many point sources of light all along the length of the slit.
        \item Consider two specific rays $r_1$ and $r_2$ emanating from the slit in same direction, one at the top and one in the middle. We know that if they are oriented at an angle that makes $\Delta r=\lambda/2$, then they cancel out.
        \item Generalizing, if any two rays satisfy $\frac{a}{2}\sin\theta=\lambda/2$ (i.e., satisfy $a\sin\theta=\lambda$), then they will cancel.
        \item Indeed, every $\theta$ satisfying $a\sin\theta=\lambda$ will cancel: Consider all the rays originating from every point in the slit that point in the $\theta$-direction, and notice that for any point in the slit, there will be a point $a/2$ units away from it; the rays from these two points will cancel. Thus, every ray is associated with another ray that cancels it out, guaranteeing that $\theta$ is an interference minimum.
        \item Note that if $\theta$ yields an interference minimum, then $\theta$ satisfying $a\sin\theta=m\lambda$ where $m\in\N$ will yield interference minima.
    \end{itemize}
\end{itemize}



\section{Intensity of Single Slit Diffraction}
\begin{itemize}
    \item Like before, $\theta=\ang{0}$ gives a \textbf{central diffraction maximum}.
    \item Finding the intensity maxima in general:
    \begin{figure}[h!]
        \centering
        \begin{tikzpicture}[
            every node/.style={black}
        ]
            \footnotesize
            \draw (0,-3) -- (0,-1) (-0.1,0) -- (0.1,0) (0,1) -- (0,3);
            \draw [very thin,dashed] (0,0) coordinate (O) -- (1,0) coordinate (N);

            \draw [semithick,->] (-0.3,0) -- ++(0,0.7) node[left]{$y$};

            \draw [rex,thick,-latex] (0,0.7) -- node[above]{$r$} ++(2,1);
            \draw [rex,thick,-latex] (0,0) -- node[above]{$r_0$} ++(2,1) coordinate (r0);

            \draw [very thin,|-|] (-0.7,-1) -- node[left]{$a$} ++(0,2);
            \pic [draw,angle eccentricity=1.3,pic text={$\theta$}] {angle=N--O--r0};
        \end{tikzpicture}
        \caption{Finding diffraction maxima.}
        \label{fig:diffMaxima}
    \end{figure}
    \begin{itemize}
        \item To find the intensity maxima, we derive an equation for the intensity in general as a function of $\theta$.
        \item To do so, we sum up every infinitesimal contribution of all the points along the slit with an integral, as follows.
        \item Since $\Delta r=y\sin\theta$ (see Figure \ref{fig:diffMaxima}), the wave function for the electric field wave along the arbitrary ray $r$ a distance $y$ from the central ray is given by
        \begin{align*}
            \cos(kr-\omega t) &= \cos(k[r_0+\Delta r]-\omega t)\\
            &= \cos\left( kr_0-\omega t+\frac{2\pi}{\lambda}\cdot y\sin\theta \right)
        \end{align*}
        \item It follows that the electric field $E$ at some point $P$ on the screen is given by
        \begin{align*}
            E &= A\int_{-a/2}^{a/2}\cos\left( kr_0-\omega t+\frac{2\pi\sin\theta}{\lambda}\cdot y \right)\dd{y}\\
            &= \frac{C}{\sin\theta}\cos(kr_0-\omega t)\sin\left( \frac{\pi a}{\lambda}\sin\theta \right)
        \end{align*}
        where $C$ represents a bunch of constants.
        \item Thus, since $I\propto E^2$,
        \begin{equation*}
            I \propto \frac{\sin^2\left( \frac{\pi a}{\lambda}\sin\theta \right)}{\sin^2\theta}
        \end{equation*}
        \item Additionally, if we define $\alpha=\frac{\pi a}{\lambda}\sin\theta$, then
        \begin{equation*}
            I = I_\text{max}\frac{\sin^2\alpha}{\alpha^2}
        \end{equation*}
        \item Notice that $\theta\to 0$ implies $\alpha\to 0$ implies $\sin(\alpha)/\alpha\to 1$ implies $I\to I_\text{max}$, as expected.
        \item Furthermore, since $\sin\alpha$ is bounded but $\alpha$ is not, $\sin^2(\alpha)/\alpha^2$ yields a graph of maxima that drop off in intensity as $\alpha\to\pm\infty$.
    \end{itemize}
    \item \textbf{Diffraction}: Bending of a light wave as it goes through a small slit.
    \begin{itemize}
        \item As slit width $a$ decreases, minima spread out.
    \end{itemize}
\end{itemize}



\section{Combining Interference and Diffraction}
\begin{itemize}
    \item \marginnote{8/19:}Every place you have a diffraction minimum, the wave that gets to the screen has 0 amplitude.
    \begin{itemize}
        \item If you have a point $P$ that's a diffraction minimum of both slits, interference doesn't matter --- you're going to have no intensity at $P$.
    \end{itemize}
    \item Slits $S_1$ and $S_2$ have the same diffraction pattern, just shifted by $d$.
    \begin{itemize}
        \item But if the diffraction pattern is large relative to $d$, as it usually is, we can neglect the shift.
    \end{itemize}
    \item Total intensity:
    \begin{figure}[h!]
        \centering
        \begin{tikzpicture}[
            every node/.append style={black}
        ]
            \footnotesize
            \begin{scope}[yshift=2cm]
                \draw [-stealth] (0,0) -- (0,2.5) node[above]{$I(\theta)$};
                \draw [stealth-stealth] (-3.5,0) -- (3.5,0) node[right]{$\theta$};
    
                \draw [orx,thick,xscale=2/pi] plot[domain=-4.33:4.33,samples=500,smooth] (\x,{2*cos(4*\x r)^2});
                \node at (1,2.4) {\normalsize 2 slits};
            \end{scope}
            \begin{scope}[yshift=-2cm]
                \draw [-stealth] (0,0) -- (0,2.5) node[above]{$I(\theta)$};
                \draw [stealth-stealth] (-3.5,0) -- (3.5,0) node[right]{$\theta$};
    
                \draw [orx,thick,xscale=2/pi] plot[domain=0.75:4.5,samples=500,smooth] (\x,{(2*cos(\x r)^2)/(\x*\x)});
                \draw [orx,thick,xscale=2/pi] plot[domain=-4.5:-0.75,samples=500,smooth] (\x,{(2*cos(\x r)^2)/(\x*\x)});
                \node at (1,2.4) {\normalsize 1 slit};
            \end{scope}
    
            \begin{scope}[xshift=10cm]
                \draw [-stealth] (0,0) -- (0,2.5) node[above]{$I(\theta)$};
                \draw [stealth-stealth] (-3.5,0) -- (3.5,0) node[right]{$\theta$};
    
                \draw [orx,thick,xscale=2/pi] plot[domain=-4.5:-0.75,samples=500,smooth] (\x,{(2*cos(\x r)^2)/(\x*\x)*cos(4*\x r)^2});
                \draw [orx,thick,xscale=2/pi] plot[domain=0.75:4.5,samples=500,smooth] (\x,{(2*cos(\x r)^2)/(\x*\x)*cos(4*\x r)^2});
                \draw [orx,thick,dashed,xscale=2/pi] plot[domain=0.75:4.5,samples=500,smooth] (\x,{(2*cos(\x r)^2)/(\x*\x)});
                \draw [orx,thick,dashed,xscale=2/pi] plot[domain=-4.5:-0.75,samples=500,smooth] (\x,{(2*cos(\x r)^2)/(\x*\x)});
            \end{scope}
    
            \draw [help lines,->,shorten >=5mm] (4,2) -- (6.5,0);
            \draw [help lines,->,shorten >=5mm] (4,-2) -- (6.5,0);
        \end{tikzpicture}
        \caption{Intensity considering both interference and diffraction.}
        \label{fig:intensityInterferenceDiffraction}
    \end{figure}
    \begin{itemize}
        \item We have that $I(\theta)=I(\theta)_\text{2 slits' interference}\times\text{diffraction envelope}$.
    \end{itemize}
\end{itemize}



\section{Circular Hole}
\begin{itemize}
    \item Consider light passing through a circular hole of diameter $a$.
    \begin{itemize}
        \item This yields concentric rings of intensity separated by nodes, i.e., a slit diffraction pattern that accounts for slits of every angle added on top of each other.
        \item As $a$ gets smaller, the central diffraction maximum gets bigger.
    \end{itemize}
    \item \textbf{Slit}: A one-dimensional opening.
    \item \textbf{Aperture}: A two-dimensional opening.
    \item We can no longer use $a\sin\theta=m\lambda$; we have to consider what happens with the extra dimensions.
    \begin{itemize}
        \item When we redo the calculation in two dimension, we get
        \begin{equation*}
            a\sin\theta = 1.22m\lambda
        \end{equation*}
        for $m\in\Z$.
        \item If $\theta$'s are small, then $\theta_\text{1st min}\approx 1.22\lambda/a$.
        \item Recall that $\theta_\text{1st min}=\theta_\text{$\frac{1}{2}$ width of central max}$, so we can use this formula to estimate the width of the central maximum.
    \end{itemize}
    \item As light passes through your pupil (an aperture), it undergoes diffraction and gets bigger before impinging on your retina.
    \begin{figure}[h!]
        \centering
        \begin{tikzpicture}
            \footnotesize
            \draw
                (3,0) -- (5,1) -- (3,2)
                (3.55,1) ellipse (3mm and 7mm)
                (4.5,0.75) to[out=120,in=-120] (4.5,1.25)
            ;
            \filldraw [draw=blx,very thick,fill=blz] (3.55,1) coordinate (pupil) ellipse (2pt and 5pt);
    
            \node (S1) [circle,fill=pix,inner sep=1.5pt,label={left:$S_1$}] at (0,1.6) {}
                edge [orx,thick] ($(S1)!1.25!(pupil)$)
            ;
            \node (S2) [circle,fill=pix,inner sep=1.5pt,label={left:$S_2$}] at (0,1.1) {}
                edge [orx,thick] ($(S2)!1.25!(pupil)$)
            ;
    
            \pic [draw,angle radius=2cm,angle eccentricity=1.15,pic text={$\theta_\text{sep}$}] {angle=S1--pupil--S2};
        \end{tikzpicture}
        \caption{Distinguishing sources of light.}
        \label{fig:lightRetina}
    \end{figure}
    \begin{itemize}
        \item Thus, to be able to distinguish two sources of light, we require $\theta_\text{sep}\geq\theta_\text{$\frac{1}{2}$ width of central max}$.
        \item For this reason, bigger telescopes are used not only to collect more light but also to minimize the effects of diffraction.
    \end{itemize}
    \item \textbf{Rayleigh criterion}: The condition for distinguishing sources of light, given by
    \begin{equation*}
        \theta_\text{sep} \geq 1.22\lambda/a
    \end{equation*}
    \item Pinhole camera.
    \begin{itemize}
        \item For a sharp focus, you want a small pinhole, improving the geometry.
        \item But if you make it too small, diffraction will come into play.
    \end{itemize}
\end{itemize}




\end{document}