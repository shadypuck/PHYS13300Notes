\documentclass[../notes.tex]{subfiles}

\pagestyle{main}
\renewcommand{\chaptermark}[1]{\markboth{\chaptername\ \thechapter\ (#1)}{}}
\setcounter{chapter}{31}

\begin{document}




\chapter{Electromagnetic Waves}
\section{Creating EM Waves}
\begin{itemize}
    \item \marginnote{8/10:}Quizzes and tests are open notebook, open notes, open textbook.
    \begin{itemize}
        \item You can use a TI-84 type calculator, but nothing fancier.
    \end{itemize}
    \item Reviews that a charge at rest generates an electric field and that a charge moving with constant velocity $v$ generates a magnetic field in addition to its electric field.
    \begin{itemize}
        \item Relativity says that you can't tell whether a charge is moving relative to you or whether you're moving relative to the charge, so a charge at rest and a moving charge have identical field lines, when appropriate frames of reference are taken.
    \end{itemize}
    \item However, when a charge accelerates for a brief time, kinks will be generated in the field lines that correspond to exactly what was going on during the acceleration.
    \begin{figure}[h!]
        \centering
        \begin{tikzpicture}
            \footnotesize
            \foreach \r in {30,90,...,330} {
                \draw [orx,semithick,postaction={decorate},decoration={markings,mark=at position 0.3 with \arrow{latex}}] (0,0) -- ++(\r:1) -- ++(-0.2,0) -- ++(\r:1);
            }
            \draw [orx,dashed,semithick] (0,0) circle (1cm);
    
            \fill [pix] (0,0) circle (2pt) node[below right,black,yshift=-3pt]{$q$};
    
            \draw [blx,ultra thick,-latex] (1.3,0) -- node[above,black]{$a$} ++(1,0);
    
            \path (-2.5,0) -- (2.5,0);
        \end{tikzpicture}
        \caption{An accelerating charge.}
        \label{fig:acceleratingCharge}
    \end{figure}
    \begin{itemize}
        \item The field lines are radial before and after the acceleration, but not radial during it (these are the kinks).
        \item The kinks are perpendicular to the field lines, and generate a transverse electric field.
        \item The transverse electric field then propagates at the speed of light.
    \end{itemize}
\end{itemize}



\section{Defining EM Waves}
\begin{itemize}
    \item \textbf{Electromagnetic wave}: The propagation of the transverse electric field.
    \item According to Faraday's Law, changing magnetic fields induce changing electric fields.
    \item According to Ampere's Law, currents induce magnetic fields.
    \begin{itemize}
        \item Maxwell adjusts this.
        \item You can have currents that are real, but also currents that are not technically currents.
        \begin{figure}[H]
            \centering
            \begin{tikzpicture}
                \footnotesize
                \draw [semithick]
                    (-1.5,0) -- (-0.3,0) node[right=1pt]{$\vec{E}$} (-0.3,-0.5) -- ++(0,1) node[above]{$+Q$}
                    (1.5,0) -- (0.3,0) (0.3,-0.5) -- ++(0,1) node[above]{$-Q$}
                ;
        
                \draw [rex,very thick,-latex] (-1.4,0.2) -- node[above,black]{$I$} ++(0.7,0);
                \draw [rex,very thick,-latex] (0.7,0.2) -- node[above,black]{$I$} ++(0.7,0);
            \end{tikzpicture}
            \caption{Generating a displacement current.}
            \label{fig:displacementCurrent}
        \end{figure}
        \begin{itemize}
            \item For example, between the plates of a charging capacitor, the changing electric field still produces a magnetic field.
            \item This \textbf{displacement current} is induced by the changing electric flux $\dv*{\Phi_E}{t}$, which is the root cause of the magnetic field.
            \item Mathematically, $I_\text{displacement}=\epsilon_0\cdot\dv*{\Phi_E}{t}$.
        \end{itemize}
        \item Thus, according to the Maxwell-Ampere law, changing electric fields induce changing magnetic fields.
    \end{itemize}
    \item As the transverse electric field approaches you, the changing magnetic field induces a planar displacement current going in one direction at the front end and the opposite direction at the other end.
    \begin{figure}[h!]
        \centering
        \begin{tikzpicture}[
            every node/.append style={black}
        ]
            \footnotesize
            \draw [-stealth] (0,0) -- (4.5,0) node[right]{$x$};
    
            \draw [rex,thick,-latex] (1.6,-1.5) -- ++(0,3) node[right]{$I_d$};
            \draw [rex,thick,-latex] (0.4,1.5) -- ++(0,-3) node[left]{$I_d$};
    
            \draw [orx,thick,-latex] (0.5,0) -- ++(0,1);
            \draw [orx,thick,-latex] (1,0) -- ++(0,1) node[above]{$E_\text{trans}$};
            \draw [orx,thick,-latex] (1.5,0) -- ++(0,1);
            \draw [white,ultra thick,shorten <=2pt] (0.5,0,0) -- ++(0,0,1);
            \draw [blx,thick,-latex] (0.5,0,0) -- ++(0,0,1);
            \draw [blx,thick,-latex] (1,0,0) -- ++(0,0,1) node[below,fill=white,inner sep=2pt]{$B_\text{trans}$};
            \draw [blx,thick,-latex] (1.5,0,0) -- ++(0,0,1);
    
            \draw [grx,ultra thick,-latex] (1.8,0.5) -- node[above]{$v$} ++(1,0);
        \end{tikzpicture}
        \caption{An idealized wave with a planar wavefront.}
        \label{fig:planarWavefront}
    \end{figure}
    \begin{itemize}
        \item As the front passes you, a magnetic field is induced, too, by the Maxwell-Ampere law.
        \item The two interchanging pulses create a self-sustaining wave.
    \end{itemize}
    \item Essentially, Maxwell's conclusion is that changing transverse electric fields and changing transverse magnetic fields induce each other.
    \begin{itemize}
        \item The speed that this occurs at is
        \begin{equation*}
            v = \frac{1}{\sqrt{\mu_0\epsilon_0}} = \SI{3e8}{\meter\per\second}
        \end{equation*}
        \item To derive this speed, Maxwell used his observation that in one dimension, the transverse electric field obeys the equation, $\pdv[2]{E_\text{trans}}{x}=\mu_0\epsilon_0\pdv[2]{E_\text{trans}}{t}$.
        \item By comparison with the wave equation, $\mu_0\epsilon_0=\frac{1}{v^2}$, which can be solved for the above.
    \end{itemize}
\end{itemize}



\section{EM Waves in the World}
\begin{itemize}
    \item Accelerating charges are commonly seen in
    \begin{enumerate}
        \item Orbiting electrons in an atom;
        \item LC circuits. 
    \end{enumerate}
    \item For an LC circuit,
    \begin{equation*}
        f_\text{EM wave} = f_\text{LC} = \frac{1}{2\pi}\cdot\frac{1}{\sqrt{LC}}
    \end{equation*}
    \item Homemade circuits can make frequencies between $\num{e4}$ - $\SI[per-mode=power]{e11}{\per\second}$, and thus since $c=f\lambda$, wavelengths between $\num{e4}$ - $\SI{e-3}{\meter}$ (waves from kilometers to millimeters in length).
    \item Orbiting electrons give frequencies between $\num{e11}$ - $\SI[per-mode=power]{e18}{\per\second}$, and wavelengths between $\num{e-3}$ - $\SI{e-10}{\meter}$ (waves from millimeters down to angstroms in length).
    \begin{itemize}
        \item Thus, we can cover 14 orders of magnitude in total.
        \item Visible is only 4000-7000 angstroms!
    \end{itemize}
\end{itemize}




\end{document}