\documentclass[../notes.tex]{subfiles}

\pagestyle{main}
\renewcommand{\chaptermark}[1]{\markboth{\chaptername\ \thechapter\ (#1)}{}}
\setcounter{chapter}{37}

\begin{document}




\chapter{Photons: Light Waves Behaving as Particles}
\section{Early Evidence for Light as a Particle}
\begin{itemize}
    \item \marginnote{8/24:}Einstein proposes the Special Theory of Relativity in 1905.
    \item Wave theory predicts that light scattered off of graphite (Compton scattering) should have the same wavelength and frequency as the incident light, but wavelength varies.
    \begin{itemize}
        \item Einstein proposes photons to account for this: Assume that photons relativistically collide with electrons and exchange energy.
        \item The momentum of the photon is proportional to the light's wavelength ($p\propto 1/\lambda$), and is equal to Planck's constant over the wavelength.
    \end{itemize}
    \item When light does something involving exchange of momentum, it behaves like a particle. When it does something involving exchange of energy, it behaves like a wave.
    \begin{itemize}
        \item Thus, light has a wave-particle duality.
    \end{itemize}
    \item Reviews bright line spectra and intro to the Bohr Model, as described in Chapter 7 of \textcite{bib:APChemNotes}.
\end{itemize}




\end{document}