\documentclass[../notes.tex]{subfiles}

\pagestyle{main}
\renewcommand{\chaptermark}[1]{\markboth{\chaptername\ \thechapter\ (#1)}{}}
\setcounter{chapter}{38}

\begin{document}




\chapter{Particles Behaving as Waves}
\section{Constructing the Quantum Model of the Atom}
\begin{itemize}
    \item \marginnote{8/24:}deBroglie: Integers come into play with waves, so why can't electrons (particles) have wavelengths?
    \begin{itemize}
        \item Rearranged $p=\hbar/\lambda$ into
        \begin{equation*}
            \lambda = \frac{\hbar}{p}
        \end{equation*}
        where $\lambda$ is the \textbf{deBroglie wavelength} and $p$ is momentum.
    \end{itemize}
    \item Thus, the Bohr model posits that the angular momentum of electrons is quantized (see Figure 7.7 from \textcite{bib:APChemNotes}).
    \begin{itemize}
        \item Specifically, $n\lambda=2\pi r_n$, i.e., some multiple of the wavelength equals the circumference of an orbit.
        \item To mathematically prevent collapsing atoms, posit a ground state described by $\lambda=2\pi r_1$.
    \end{itemize}
\end{itemize}




\end{document}