\documentclass[../notes.tex]{subfiles}

\pagestyle{main}
\renewcommand{\chaptermark}[1]{\markboth{\chaptername\ \thechapter\ (#1)}{}}
\setcounter{chapter}{33}

\begin{document}




\chapter{Geometric Optics}
\section{Mirrors}
\begin{itemize}
    \item \marginnote{8/12:}Take what we know about reflection and refraction and apply it to mirrors and lenses.
    \item Bathroom mirror: A flat piece of glass with a shiny background behind it.
    \item Spherical mirror: Some part of a reflective sphere.
    \item Relating the \textbf{radius of curvature} and the distance to the \textbf{focal point}, or the \textbf{focal length}, of a concave spherical mirror.
    \begin{figure}[h!]
        \centering
        \begin{tikzpicture}[
            scale=1.3,
            pics/point/.style args={#1:#2}{code={
                \coordinate (#2);
                \node [circle,fill=pix,inner sep=1.5pt,label={#1:$#2$}] {};
            }}
        ]
            \footnotesize
            \def\r{30}
            \draw [name path=axis] (0,0) -- (4,0) node[right]{axis};
    
            \draw [blx,thick] (2.5,1) arc[start angle=45,end angle=-45,radius={2^0.5}];
    
            \draw [very thin,dashed] (1.5,0) pic{point=below:C} -- ++(\r:2^0.5) pic{point=right:A};
            \pic at ({1.5+2^0.5},0) {point=below right:V};
    
            \draw [
                rex,thick,-latex,name path=ray,postaction={decorate},decoration={
                    markings,mark=at position 0.1 with \arrow{latex}
                }
            ]
                (A) ++(-2.7,0) coordinate (1)
                -- (A) node[circle,fill=pix,inner sep=1.5pt]{}
                -- ++({180+2*\r}:2)
            ;
    
            \path [name intersections={of=axis and ray,by=F}] pic at (F) {point=below:F};
            \pic [draw,angle eccentricity=1.3,pic text={$\theta$}] {angle=1--A--C};
            \pic [draw,angle eccentricity=1.3,pic text={$\theta$}] {angle=C--A--F};
            \pic [draw,angle eccentricity=1.3,pic text={$\theta$}] {angle=F--C--A};
        \end{tikzpicture}
        \caption{Spherical mirror analysis.}
        \label{fig:sphericalMirror}
    \end{figure}
    \begin{itemize}
        \item For a distant source $s$, we can approximate the rays as parallel.
        \item Consider one specific ray.
        \item Drawing a normal to the surface of the spherical mirror, this normal will be a radius passing through the \textbf{center of curvature}.
        \item It follows by the law of reflection and the alternate interior angle theorem that all $\theta$ are equal.
        \item This makes $\triangle CFA$ isosceles.
        \item Now assume \textbf{paraxial rays}.
        \item Then $\triangle CFA$ converges to a line segment (a radius) with midpoint $F$.
        \item Therefore,
        \begin{equation*}
            f = \frac{R}{2}
        \end{equation*}
        for a concave spherical mirror.
    \end{itemize}
    \item \textbf{Radius of curvature}: The radius of the sphere into which the spherical mirror would fit. \emph{Denoted by} $\bm{R}$.
    \item \textbf{Focal point}: The point where parallel rays converge, after reflecting off of a curved mirror. \emph{Denoted by} $\bm{F}$.
    \item \textbf{Focal length}: The distance from the focal point to the mirror. \emph{Denoted by} $\bm{f}$.
    \item \textbf{Center of curvature}: The center of the sphere into which the spherical mirror would fit. \emph{Denoted by} $\bm{C}$.
    \item \textbf{Paraxial ray}: A ray that is close to the mirror axis.
    \item Alternatively, if you look at a convex mirror, it appears (via ray tracing) that all rays of light originated from the focal point.
    \begin{itemize}
        \item The image in this type of mirror will be a virtual image.
        \item In this case, we say that $f=-R/2$.
    \end{itemize}
    \item A spherical mirror does not focus all rays to a \emph{single} point --- the rays only go to \emph{approximately} the same point.
    \begin{itemize}
        \item The farther the rays get from being paraxial, the more they diverge from the focal point.
    \end{itemize}
    \item To have a true focal point, we need a parabolic mirror.
    \item Unfortunately, parabolic mirrors are hard to make.
    \item \textbf{Ray tracing}: Take a few \textbf{principal rays} and see where the image forms.
    \begin{figure}[h!]
        \centering
        \begin{tikzpicture}[
            scale=2.5,
            pics/point/.style args={#1:#2}{code={
                \coordinate (#2);
                \node [circle,fill=pix,inner sep=1.5pt,label={#1:${\color{black}#2}$}] {};
            }}
        ]
            \footnotesize
            \draw [name path=axis] (0,0) -- (3.5,0) node[right]{axis};
    
            \draw [blx,thick,name path=mirror] (2,1) arc[start angle=45,end angle=-45,radius={2^0.5}];
    
            \draw [grx,ultra thick,-stealth] (0.5,0) node[above left,black]{object} edge [black,thin,->,out=90,in=180,shorten >=2pt] (0.5,0.4) -- ++(0,{2^0.5/2}) coordinate (P1);
    
            \path [name path=Afinder] (P1) -- ++(4,0);
            \draw [
                rex,thick,-latex,postaction={decorate},
                decoration={markings,mark=at position 0.2 with \arrow{latex}},
                name intersections={of=mirror and Afinder,by=A},name path=parallel
            ] (P1) -- (A) -- ++(-120:2);
            \draw [
                rex,thick,-latex,postaction={decorate},
                decoration={markings,mark=at position 0.2 with \arrow{latex}},
                name intersections={of=mirror and axis,by=V},name path=central
            ] (P1) -- (V) pic{point={[xshift=-2pt,yshift=2pt]below right:V}} -- (P1 |- 0,{-2^0.5/2});
            \path [name intersections={of=axis and parallel,by=F},name path=Fmirror] (P1) -- ($(P1)!1.5!(F)$);
            \draw [
                rex,thick,-latex,line join=bevel,postaction={decorate},
                decoration={markings,mark=at position 0.2 with \arrow{latex}},
                name intersections={of=mirror and Fmirror}
            ] (P1) -- (intersection-1) -- ++(-2,0);
            \path [name intersections={of=parallel and central,by={,,I}},name path=Imirror] (P1) -- ($(P1)!2!(I)$);
            \draw [
                rex,thick,postaction={decorate},
                decoration={
                    markings,
                    mark=at position 0.33 with \arrow{latex},
                    mark=at position 0.85 with \arrowreversed{latex}
                },
                name intersections={of=axis and Imirror,by=C},name intersections={of=mirror and Imirror}
            ] (P1) -- (intersection-1) pic at (C) {point={[xshift=2pt,yshift=2pt]below left:C}};
    
            \draw [grx,ultra thick,stealth-] (I) node[below left,yshift=-1mm,black,fill=white,inner sep=1.5pt]{image} edge[black,thin,->,out=90,in=180,shorten >=2pt] (I |- 0,-0.1) -- (I |- 0,0);
            \pic at (F) {point={[fill=white,circle,inner sep=0pt,xshift=1pt,yshift=0pt]below right:F}};
        \end{tikzpicture}
        \caption{Ray tracing a concave mirror.}
        \label{fig:rayTracing}
    \end{figure}
    \item \textbf{Principal ray}: A ray with well-understood geometry.
    \item \textbf{Focal ray}: A ray that goes through the focal point.
    \begin{itemize}
        \item Will be reflected out as a parallel ray.
    \end{itemize}
    \item \textbf{Central ray}: A ray that hits the center of the mirror.
    \begin{itemize}
        \item Will be reflected out with the same incident angle relative to the axis.
    \end{itemize}
    \item \textbf{Radial ray}: A ray that passes through the center of curvature.
    \begin{itemize}
        \item It follows a \emph{radius} of the sphere of curvature.
        \item Will be reflected such that it heads right back to where it started.
    \end{itemize}
    \item Calculating the location of the image in a spherical mirror.
    \begin{figure}[h!]
        \centering
        \begin{tikzpicture}[
            scale=2.5,
            pics/point/.style args={#1:#2}{code={
                \coordinate (#2);
                \node [circle,fill=pix,inner sep=1.5pt,label={#1:${\color{black}#2}$}] {};
            }}
        ]
            \footnotesize
            \draw [name path=axis] (0,0) -- (3.5,0) node[right]{axis};
    
            \draw [blx,thick,name path=mirror] (2,1) arc[start angle=45,end angle=-45,radius={2^0.5}];
    
            \draw [grx,ultra thick,-stealth] (0.5,0) coordinate (Q1) node[below left,black]{$Q$} node[above left,black]{object} edge [black,thin,->,out=90,in=180,shorten >=2pt] (0.5,0.4) -- ++(0,{2^0.5/2}) coordinate (P1) node[above left,black]{$P$};
    
            \path [name path=Afinder] (P1) -- ++(4,0);
            \path [name intersections={of=mirror and Afinder,by=A},name path=parallel] (P1) -- (A) -- ++(-120:2);
            \draw [
                rex,thick,-latex,postaction={decorate},
                decoration={markings,mark=at position 0.2 with \arrow{latex}},
                name intersections={of=mirror and axis,by=V},name path=central
            ] (P1) -- (V) pic{point={[xshift=-2pt,yshift=2pt]below right:V}} -- (P1 |- 0,{-2^0.5/2});
            \path [name intersections={of=parallel and central,by={,,I}},name path=Imirror] (P1) -- ($(P1)!2!(I)$);
            \draw [
                rex,thick,postaction={decorate},
                decoration={
                    markings,
                    mark=at position 0.33 with \arrow{latex},
                    mark=at position 0.85 with \arrowreversed{latex}
                },
                name intersections={of=axis and Imirror,by=C},name intersections={of=mirror and Imirror}
            ] (P1) -- (intersection-1) pic at (C) {point={[xshift=2pt,yshift=2pt]below left:C}};
    
            \draw [grx,ultra thick,stealth-] (I) coordinate (P2) node[below=1pt,xshift=-1pt,black]{$P'$} node[below right,xshift=4mm,black]{image} edge[black,thin,->,out=135,in=0,shorten >=2pt] (I |- 0,-0.1) -- (I |- 0,0) coordinate (Q2) node[above,black]{$Q'$};
    
            \path [name intersections={of=axis and parallel,by=F}] (F) pic {point={[fill=white,circle,inner sep=0pt,xshift=1pt,yshift=0pt]below right:F}};
            \pic [draw,angle radius=2mm] {right angle=C--Q1--P1};
            \pic [draw,angle radius=2mm] {right angle=C--Q2--P2};
            \pic [draw,angle radius=4mm,angle eccentricity=1,pic text={=},pic text options={rotate=160,yshift=-0.5pt}] {angle=P1--C--Q1};
            \pic [draw,angle radius=4mm,angle eccentricity=1,pic text={=},pic text options={rotate=-20,yshift=-0.5pt}] {angle=P2--C--Q2};
            \pic [draw,angle radius=6mm,angle eccentricity=1,pic text={\large--},pic text options={rotate=170}] {angle=P1--V--Q1};
            \pic [draw,angle radius=6mm,angle eccentricity=1,pic text={\large--},pic text options={rotate=190}] {angle=Q2--V--P2};
    
            \draw [very thin,|-|] (P1) ++(0,0.4) -- node[above]{$s$} ++(1.92,0);
            \draw [very thin,|-|] (P2) ++(0,-0.8) -- node[below]{$s'$} ++(0.76,0);
        \end{tikzpicture}
        \caption{Deriving the mirror equation.}
        \label{fig:mirrorEquation}
    \end{figure}
    \begin{itemize}
        \item The image in a spherical mirror will be an inverted, \textbf{real} image.
        \item Two principal rays define where the image is.
        \item Considering the image of the top point in our object, choose to analyze a central ray and a radial ray.
        \item Then $\triangle PVQ$ and $\triangle P'VQ'$ are similar, and $\triangle PCQ$ and $\triangle P'CQ'$ are similar.
        \item It follows that
        \begin{align*}
            \frac{\overline{PQ}}{{\color{white}(}\overline{P'Q'}{\color{white})}} &= \frac{\overline{QV}}{{\color{white}(}\overline{Q'V}{\color{white})}} = \frac{s}{s'}&
            \frac{\overline{PQ}}{{\color{white}(}\overline{P'Q'}{\color{white})}} &= \frac{\overline{QC}}{{\color{white}(}\overline{Q'C}{\color{white})}} = \frac{s-R}{R-s'}
        \end{align*}
        \item Thus, we have the following, which can be solved for the mirror equation.
        \begin{equation*}
            \frac{s}{s'} = \frac{s-R}{R-s'}
        \end{equation*}
    \end{itemize}
    \item \textbf{Real image}: An image that could be substituted for an image on a screen in real space.
    \item \textbf{Object distance}: The distance from an object to a spherical mirror. \emph{Denoted by} $\bm{s}$.
    \item \textbf{Image distance}: The distance from an object's image to the mirror. \emph{Denoted by} $\bm{s'}$.
    \item \textbf{Mirror equation}: The formula
    \begin{equation*}
        \frac{1}{s}+\frac{1}{s'} = \frac{1}{f}
    \end{equation*}
    \item \textbf{Lateral magnification}: The ratio of the distance from the object to the mirror and the distance from the image to the mirror. \emph{Denoted by} $\bm{m}$.
    \begin{itemize}
        \item Mathematically,
        \begin{equation*}
            m = \frac{-s'}{s}
        \end{equation*}
        \item If $m<0$, the image is inverted.
        \item If $m>0$, the image is upright.
        \item Sign convention: $s,s'$ are positive in front of the mirror and negative behind the mirror.
    \end{itemize}
    \item Check the above equations on the example of a plane mirror.
    \begin{itemize}
        \item For a plane mirror, $R=\infty$, so $\frac{1}{f}=0$.
        \item It follows by the mirror equation that $s'=-s$.
        \item Therefore, $m=1$, as desired.
    \end{itemize}
    \item \marginnote{8/13:}Ray tracing a convex mirror.
    \begin{figure}[H]
        \centering
        \begin{tikzpicture}[
            scale=2.5,
            pics/point/.style args={#1:#2}{code={
                \coordinate (#2);
                \node [circle,fill=pix,inner sep=1.5pt,label={#1:${\color{black}#2}$}] {};
            }}
        ]
            \footnotesize
            \draw [name path=axis] (0,0) -- (3.5,0) node[right]{axis};
    
            \draw [blx,thick,name path=mirror] (2,1) arc[start angle=135,end angle=225,radius={2^0.5}];
    
            \draw [grx,ultra thick,-stealth] (0.5,0) node[above left,black]{object} edge [black,thin,->,out=90,in=180,shorten >=2pt] (0.5,0.4) -- ++(0,{2^0.5/2}) coordinate (P1);
    
            \path [name path=Afinder] (P1) -- ++(4,0);
            \draw [
                rex,thick,-latex,postaction={decorate},
                decoration={markings,mark=at position 0.2 with \arrow{latex}},
                name intersections={of=mirror and Afinder,by=A},name path=parallel
            ] (P1) -- (A) -- ++(120:1);
            \draw [
                rex,thick,-latex,postaction={decorate},
                decoration={markings,mark=at position 0.2 with \arrow{latex}},
                name intersections={of=mirror and axis,by=V},name path=central
            ] (P1) -- (V) -- (P1 |- 0,{-2^0.5/2});
            \path [name path=Ffinder] (A) -- ++(-60:1);
            \draw [rex,thick,dashed,name intersections={of=Ffinder and axis,by=F}] (A) -- (F) pic{point=below:F};
            \path [name path=P2finder] (V) -- ++($(V)-(P1)+(0,2^0.5)$);
            \draw [rex,thick,dashed,name intersections={of=P2finder and Ffinder,by=P2}] (V) -- (P2);
    
            \draw [grx,ultra thick,-stealth] (P2 |- 0,0) -- (P2) node[above right,black]{image} edge [out=-90,in=0,->,black,thin,shorten >=2pt] (P2 |- 0,0.1);
        \end{tikzpicture}
        \caption{Ray tracing a convex mirror.}
        \label{fig:rayTracingConvex}
    \end{figure}
    \begin{itemize}
        \item Upright, smaller, virtual image.
    \end{itemize}
\end{itemize}



\section{Lenses}
\begin{itemize}
    \item \marginnote{8/12:}\textbf{Double convex} (lens): A lens with two convex exterior surfaces. \emph{Also known as} \textbf{convex} (lens), \textbf{converging} (lens), \textbf{positive} (lens).
    \item \textbf{Double concave} (lens): A lens with two concave exterior surfaces. \emph{Also known as} \textbf{concave} (lens), \textbf{diverging} (lens), \textbf{negative} (lens).
    \item \textbf{Plano convex} (lens): A lens with one concave exterior surface and one flat exterior surface.
    \begin{figure}[h!]
        \centering
        \begin{tikzpicture}[scale=1.3]
            \footnotesize
            \draw [name path={axis}] (0,0) -- (6,0) node[right]{axis};
    
            \draw [very thin,dashed]
                (1,0) coordinate (C) node[circle,fill=pix,inner sep=1.5pt,label={below:$C$}]{}
                ++(30:-0.5) -- ++(30:0.5) -- node[below]{$R$} ++(30:{2^0.5}) coordinate (V)
                -- ++(30:1) coordinate (B)
                (V) -- ++(1.5,0) coordinate (D)
            ;
    
            \draw [rex,thick,postaction={decorate},decoration={markings,mark=at position 0.1 with \arrow{latex}},-latex,name path={ray}] (V) ++(-2.2,0) coordinate (A) -- (V) -- ++(3.5,-1.3);
            
            \draw [blx,thick] (2,1) arc[start angle=45,end angle=-45,radius={2^0.5}] -- node[below left,black,text height=1.5ex,text depth=0.25ex]{$1$} node[below right,black,text height=1.5ex,text depth=0.25ex]{$n$} node[below right,xshift=5mm,black,text height=1.5ex,text depth=0.25ex]{$1$} cycle;
    
            \path [name intersections={of=axis and ray,by=F}] (F) node [circle,fill=pix,inner sep=1.5pt,label={below:$F$}] {};
            \draw [very thin,|-|] (-0.15,0) -- node[left]{$y$} (-0.15,0 |- V);
            \draw [very thin,|-|] (V |- 0,1.3) -- node[above]{$f$} (F |- 0,1.3);
            \pic [draw,angle eccentricity=1.3,angle radius=6mm,pic text={$\theta_1$}] {angle=A--V--C};
            \pic [draw,angle eccentricity=1.2,angle radius=1cm,pic text={$\theta_1$}] {angle=D--V--B};
            \pic [draw,angle eccentricity=1.3,angle radius=1cm,pic text={$\theta_\text{def}$}] {angle=F--V--D};
            \pic [draw,angle eccentricity=1.41,angle radius=5mm,pic text={$\theta_2$},pic text options={fill=white,inner sep=1.5pt}] {angle=F--V--B};
        \end{tikzpicture}
        \caption{Plano convex lens analysis.}
        \label{fig:planoConvexLens}
    \end{figure}
    \begin{itemize}
        \item Clearly, $\theta_\text{def}=\theta_2-\theta_1$ and $\sin\theta_1=y/R$.
        \item If this is a \textbf{thin lens}, then we can invoke the small angle approximation and say that $\theta_1\approx y/R$.
        \item By Snell's Law, $\sin\theta_2=n\sin\theta_1$.
        \item Invoking the SAA again, we have that $\theta_2=n\theta_1$.
        \item It follows that
        \begin{align*}
            \theta_\text{def} &= \theta_2-\theta_1\\
            &= n\theta_1-\theta_1\\
            &= (n-1)\theta_1\\
            &= (n-1)\cdot\frac{y}{r}
        \end{align*}
        \item Therefore, since $\sin\theta_\text{def}=\frac{y}{f}$ (i.e., with the SAA $\theta_\text{def}=y/f$), we have that
        \begin{align*}
            f &= \frac{y}{\theta_\text{def}}\\
            &= \frac{y}{(n-1)\cdot y/R}\\
            &= \frac{R}{n-1}
        \end{align*}
        \item More commonly, we express this with
        \begin{equation*}
            \frac{1}{f} = (n-1)\cdot\frac{1}{R}
        \end{equation*}
    \end{itemize}
    \item \textbf{Thin lens}: A lens where $\theta$'s are small.
    \item If you substitute a convex lens for the plano convex lens, everything gets doubled.
    \begin{itemize}
        \item Importantly, we now have
        \begin{equation*}
            \frac{1}{f} = (n-1)\cdot\frac{2}{R}
        \end{equation*}
    \end{itemize}
    \item We can also generalize a convex lens to a lens with two convex sides of varying radii of curvature on its two sides.
    \item \textbf{Lens maker's equation}: The following formula.
    \begin{equation*}
        \frac{1}{f} = (n-1)\cdot\left( \frac{1}{R_1}+\frac{1}{R_2} \right)
    \end{equation*}
    \begin{itemize}
        \item Notice that if, for instance, the first side is planar, then $R_1=\infty$ and the $1/R_1$ term disappears.
        \item Sign convention: $R_1,R_2$ are positive for convex lenses and negative for concave lenses.
    \end{itemize}
    \item Remember that knowing the focal length of a lens allows us to ray trace.
    \begin{figure}[h!]
        \centering
        \begin{tikzpicture}[
            pics/point/.style args={#1:#2}{code={
                \coordinate (#2);
                \node [circle,fill=pix,inner sep=1.5pt,label={#1:${\color{black}#2}$}] {};
            }}
        ]
            \footnotesize
            \fill [blz] (3,1) arc[start angle=30,end angle=-30,radius=2] arc[start angle=210,end angle=150,radius=2];
            \draw [name path=axis] (0,0) -- (6,0) node[right]{axis};
            \draw [blx,thick] (3,1) arc[start angle=30,end angle=-30,radius=2] arc[start angle=210,end angle=150,radius=2];
    
            \draw [grx,ultra thick,-stealth] (1,0) node[above left,black]{object} edge [black,thin,->,out=90,in=180,shorten >=2pt] (1,0.55) -- ++(0,0.9) coordinate (P1);
            \draw [grx,ultra thick,-stealth] (5,0) node[below right,black]{image} edge [black,thin,->,out=-90,in=0,shorten >=2pt] (5,-0.55) -- ++(0,-0.9) coordinate (P2);
    
            \draw [rex,thick,-latex,postaction={decorate},decoration={markings,mark=at position 0.2 with \arrow{latex}},name path=parallel] (P1) -- (3.05,0.9) -- ($(3.05,0.9)!1.5!(P2)$);
            \draw [rex,thick,-latex,postaction={decorate},decoration={markings,mark=at position 0.2 with \arrow{latex}}] (P1) -- ($(P1)!1.3!(P2)$);
            \draw [rex,thick,-latex,postaction={decorate},decoration={markings,mark=at position 0.2 with \arrow{latex}},name path=focal] (P1) -- (2.95,-0.9) -- ($(2.95,-0.9)!1.6!(P2)$);
    
            \path [name intersections={of=parallel and axis}] (intersection-1) pic{point=above right:F};
            \path [name intersections={of=focal and axis}] (intersection-1) pic{point=below left:F};
        \end{tikzpicture}
        \caption{Convex lens rays.}
        \label{fig:lensRaysConvex}
    \end{figure}
    \begin{itemize}
        \item Central rays in thin lenses have negligible lateral displacement.
        \item An image through a convex lens will be an inverted real image.
    \end{itemize}
    \item \marginnote{8/13:}Ray tracing a concave lens.
    \begin{itemize}
        \item Upright, smaller, virtual image.
    \end{itemize}
    \item Note that the lens maker's equation used by \textcite{bib:YoungFreedman} is $1/f=(n-1)(1/R_1-1/R_2)$, along with a complicated sign convention.
    \begin{itemize}
        \item The textbook uses this formula because it follows from the study of thick lenses, with which many older textbooks start.
    \end{itemize}
    \item A concave mirror and a plano convex lens with a mirrored background image the same way.
    \begin{itemize}
        \item A convex lens mirrors the same way as both, but with the image on the other side of the lens as opposed to the same side of the apparatus.
    \end{itemize}
    \item Therefore, the lens equation is also analogously
    \begin{equation*}
        \frac{1}{s}+\frac{1}{s'} = \frac{1}{f}
    \end{equation*}
    \begin{itemize}
        \item However, this equation comes with the sign convention that $s$ is positive in front of the lens and negative behind the lens, and $s'$ is positive behind the lens and negative in front of the lens.
    \end{itemize}
    \item Similarly,
    \begin{equation*}
        m = \frac{-s'}{s}
    \end{equation*}
    for lenses, with the new sign convention.
    \item Note that the focal length is positive for convex lenses and negative for concave lenses.
    \item Just like mirrors can image images from other mirrors, lenses can image images from other lenses.
    \begin{itemize}
        \item You do it the same way, too --- one at a time.
    \end{itemize}
    \item Example: Suppose that you have two identical thin lenses, $\SI{15}{\centi\meter}$ apart, with focal length $\SI{10}{\centi\meter}$ each. Place an object $\SI{15}{\centi\meter}$ to the left of the left lens. How does it image?
    \begin{figure}[h!]
        \centering
        \begin{tikzpicture}[
            pics/point/.style args={#1:#2}{code={
                \coordinate (#2);
                \node [circle,fill=pix,inner sep=1.5pt,label={#1:${\color{black}#2}$}] {};
            }},
            every node/.style={black}
        ]
            \footnotesize
            \fill [blz] (2,1) arc[start angle=30,end angle=-30,radius=2] arc[start angle=210,end angle=150,radius=2];
            \fill [blz] (3.5,1) arc[start angle=30,end angle=-30,radius=2] arc[start angle=210,end angle=150,radius=2];
            \draw [name path=axis] (0,0) -- (6,0) node[right]{axis};
            \draw [blx,thick] (2,1)   node[above]{$1$} arc[start angle=30,end angle=-30,radius=2] node[below]{$f=\SI{10}{\centi\meter}$} arc[start angle=210,end angle=150,radius=2];
            \draw [blx,thick] (3.5,1) node[above]{$2$} arc[start angle=30,end angle=-30,radius=2] node[below]{$f=\SI{10}{\centi\meter}$} arc[start angle=210,end angle=150,radius=2];
    
            \pic at (1,0) {point=below:F_1};
            \pic at (3,0) {point=below:F_1};
            \pic at (2.5,0) {point=below:F_2};
            \pic at (4.5,0) {point=below:F_2};
    
            \draw [grx,ultra thick,-stealth] (0.5,0) -- ++(0,0.8) node[above right=-2pt,black]{object} edge [black,thin,->,out=-90,in=0,shorten >=2pt] (0.5,0.4);
            \draw [grx,ultra thick,-stealth] (4.1,0) -- ++(0,-0.64) node[below right=-2pt,black]{image} edge [black,thin,->,out=90,in=0,shorten >=2pt] (4.1,-0.3);
    
            \draw (0,-1.7) -- node[below=6mm]{$\si{\centi\meter}$} ++(6,0);
            \foreach \x [evaluate=\x as \num using int(\x*10)] in {0,0.5,...,6} {
                \draw (\x,-1.6) -- ++(0,-0.2) node[below]{$\num$};
            }
        \end{tikzpicture}
        \caption{Imaging two lenses.}
        \label{fig:twoLensImage}
    \end{figure}
    \begin{itemize}
        \item Image location:
        \begin{itemize}
            \item For lens 1, we have $f=\SI{10}{\centi\meter}$ and $s=\SI{15}{\centi\meter}$, so $s'=\SI{30}{\centi\meter}$ by the lens equation.
            \item For lens 2, we still have $f=\SI{10}{\centi\meter}$ but now we have have $s=\SI{-15}{\centi\meter}$, so $s'=\SI{6}{\centi\meter}$ by the lens equation.
        \end{itemize}
        \item Magnification:
        \begin{align*}
            m_\text{total} &= m_1m_2\\
            &= \left( -\frac{\SI{30}{\centi\meter}}{\SI{15}{\centi\meter}} \right)\left( -\frac{\SI{6}{\centi\meter}}{\SI{-15}{\centi\meter}} \right)\\
            &= -\frac{4}{5}
        \end{align*}
        \item Thus, the total image is inverted, $4/5$ times the size of the original, and $\SI{6}{\centi\meter}$ to the right of the rightmost lens.
    \end{itemize}
    \item Mirrors and lenses have limitations.
    \begin{itemize}
        \item Thus, it does make sense to use multiple mirrors/lenses in some circumstances.
        \item Compensating for the paraxial approximation: When you need a sharper focus in good cameras.
        \item Compensating for the dependence of $n$ on frequency: Have any single lens do less work.
    \end{itemize}
\end{itemize}




\end{document}