\documentclass[../notes.tex]{subfiles}

\pagestyle{main}
\renewcommand{\chaptermark}[1]{\markboth{\chaptername\ \thechapter\ (#1)}{}}
\setcounter{chapter}{39}

\begin{document}




\chapter{Quantum Mechanics I: Wave Functions}
\section{The Wave Equation}
\begin{itemize}
    \item \marginnote{8/24:}\textbf{Quantum wave}: The wave-like nature of an electron.
    \item But waves must satisfy a wave equation.
    \begin{itemize}
        \item The classical one didn't work.
        \item In 1925, Schr\"{o}dinger determined that in one dimension, an electron moving in a potential $V$ (the nucleus-electron Coulombic attraction) satisfies
        \begin{equation*}
            -\frac{\hbar^2}{2m}\pdv[2]{\Psi(x,t)}{x}+V\Psi = i\hbar\pdv{\Psi}{t}
        \end{equation*}
        \begin{itemize}
            \item This wave equation wasn't derived in an analogous method to Figure \ref{fig:waveEquationDerivation}, but rather was constructed from conservation of energy.
        \end{itemize}
        \item This wave function has both real and imaginary parts with the inclusion of $i=\sqrt{-1}$.
    \end{itemize}
\end{itemize}



\section{Electrons in the Double Slit Experiment}
\begin{itemize}
    \item Electrons exhibit both diffraction and interference in the double slit experiment.
    \begin{itemize}
        \item $d\sin\theta=m\lambda$ applies when $\lambda$ is the deBroglie wavelength.
        \item Even with only one electron being emitted at a time, you get interference (further confirms wave-like nature of electrons): $\Psi_\text{electron}=\Psi_\text{slit 1}+\Psi_\text{slit 2}$.
        \item Observing which slit an electron goes through removes the interference pattern.
    \end{itemize}
    \item $P\propto\Psi^2$ is independent of time, so you have static charge distributions.
\end{itemize}




\end{document}