\documentclass[../notes.tex]{subfiles}

\pagestyle{main}
\renewcommand{\chaptermark}[1]{\markboth{\chaptername\ \thechapter\ (#1)}{}}
\setcounter{chapter}{39}

\begin{document}




\chapter{Quantum Mechanics I: Wave Functions}
\section{The Wave Equation}
\begin{itemize}
    \item \marginnote{8/24:}\textbf{Quantum wave}: The wave-like nature of an electron.
    \item But waves must satisfy a wave equation.
    \begin{itemize}
        \item The classical one didn't work.
        \item In 1925, Schr\"{o}dinger determined that in one dimension, an electron moving in a potential $V$ (the nucleus-electron Coulombic attraction) satisfies
        \begin{equation*}
            -\frac{\hbar^2}{2m}\pdv[2]{\Psi(x,t)}{x}+V\Psi = i\hbar\pdv{\Psi}{t}
        \end{equation*}
        \begin{itemize}
            \item This wave equation wasn't derived in an analogous method to Figure \ref{fig:waveEquationDerivation}, but rather was constructed from conservation of energy.
        \end{itemize}
        \item This wave function has both real and imaginary parts with the inclusion of $i=\sqrt{-1}$.
    \end{itemize}
\end{itemize}



\section{Electrons in the Double Slit Experiment}
\begin{itemize}
    \item Electrons exhibit both diffraction and interference in the double slit experiment.
    \begin{itemize}
        \item $d\sin\theta=m\lambda$ applies when $\lambda$ is the deBroglie wavelength.
        \item Even with only one electron being emitted at a time, you get interference (further confirms wave-like nature of electrons): $\Psi_\text{electron}=\Psi_\text{slit 1}+\Psi_\text{slit 2}$.
        \item Observing which slit an electron goes through removes the interference pattern.
    \end{itemize}
    \item $P\propto\Psi^2$ is independent of time, so you have static charge distributions.
\end{itemize}



\section{Office Hours (Gazes)}
\begin{itemize}
    \item Quantifying optical roughness?
    \item What are $C_V$ and $C_p$? Are they specific to certain gasses?
    \item What is an adiabatic process? Is it a straight conversion from temperature (internal) energy to work? I find this highly unintuitive.
    \begin{itemize}
        \item Heat flow is held constant, so we have the process happening in an insulating container.
        \item Temperature, pressure, and volume are all changing, hence the isotherm-crossing curve in Figures \ref{fig:ottoCycle} and \ref{fig:carnotCycle}.
        \item Running a fire piston backwards does cause the interior to get colder.
        \item Releasing air from a compressed air can is also an adiabatic process (pressure decreases, volume effectively gets much bigger as it enters the room, and can gets cold).
    \end{itemize}
    \item Additive entropies derivation.
\end{itemize}



\section{Final Review Sheet}
\begin{itemize}
    \item \marginnote{2/1/24:}A wavefunction propagates via $f(x-vt)$.
    \item Derive the wave equation via free-body diagramming an infinitesimal segment of the rope.
    \begin{equation*}
        \pdv[2]{y}{x} = \frac{M}{F}\cdot\pdv[2]{y}{t}
    \end{equation*}
    \item Convert it into the following form by substituting a generic sinusoidal wave $y(x,t)=A\sin(kx-\omega t)$ into the above wave equation and recalling that $v=\omega/k$.
    \begin{equation*}
        \pdv[2]{y}{x} = \frac{1}{v^2}\cdot\pdv[2]{y}{t}
    \end{equation*}
    \item Transverse velocity \& acceleration.
    \item Average power of a wave on a string.
    \begin{equation*}
        \bar{P} = \frac{1}{2}Mv\omega^2A^2
    \end{equation*}
    \begin{itemize}
        \item Derive with $P=\vec{F}\cdot\vec{v}$. Recall that the average value of $\cos^2(x)$ is $1/2$.
    \end{itemize}
    \item Compound string, boundary conditions, reflection and transmission (like in quantum).
    \begin{itemize}
        \item Limiting case is standing waves.
    \end{itemize}
    \item Speed of sound in terms of the medium at STP.
    \begin{equation*}
        v_\text{sound} = \sqrt{\frac{1.4p_0}{\rho_0}}
    \end{equation*}
    \begin{itemize}
        \item $p_0$ is the pressure of the air; $\rho_0$ is the density of the air.
    \end{itemize}
    \item Intensity: Power per unit area.
    \begin{equation*}
        I = \frac{\bar{P}}{4\pi r^2}
    \end{equation*}
    \item Constructive and destructive interference of soundwaves.
    \item Beating for sound waves of very similar frequencies.
    \item For the Doppler effect, recall that $v=\lambda f$ and then think about how $v$ changes.
    \begin{itemize}
        \item When an observer runs toward the source at $v_o$, the wavefronts hit them at velocity $v+v_o$, so they hear them at frequency
        \begin{equation*}
            f' = \frac{v+v_o}{\lambda}
            = \frac{v+v_o}{v}\cdot f
            > f
        \end{equation*}
        \item If the source is moving toward the observer at speed $v_s$, then the wavefronts still hit the observer at the same speed $v$ and they are still emitted at the same frequency $f$, but the spacing $\lambda$ between them changes. Relative to the source, the waves move away more slowly, so
        \begin{equation*}
            \lambda' = \frac{v-v_s}{f}
        \end{equation*}
        which implies that
        \begin{equation*}
            f' = \frac{v}{\lambda'}
            = \frac{v}{v-v_s}\cdot f
        \end{equation*}
        \item The tricky part here is the special relativity I have to use and keep straight in my head.
    \end{itemize}
    \item A things about ideal gases.
    \begin{itemize}
        \item Recall that
        \begin{align*}
            KE &= \frac{3}{2}RT = \frac{1}{2}M\bar{v}^2&
            v_\text{rms} = \sqrt{\frac{3RT}{M}}
        \end{align*}
        \item Let $p$ be pressure, $\rho$ be density, and $\bar{v}$ be the average velocity of the molecules. Then
        \begin{equation*}
            p = \frac{1}{3}\rho\bar{v}^2
        \end{equation*}
        \begin{itemize}
            \item This yields
            \begin{equation*}
                v_\text{rms} = \sqrt{\frac{3\rho}{p}}
            \end{equation*}
            \item I can relate this to the density form of the ideal gas law:
            \begin{align*}
                PV &= nRT\\
                \frac{PM}{RT} &= \frac{nM}{V} = \rho\\
                \frac{1}{3}\bar{v}^2 = \frac{RT}{M} &= \frac{P}{\rho}
            \end{align*}
        \end{itemize}
    \end{itemize}
    \item In isothermal conditions,
    \begin{equation*}
        W = -nRT\ln\frac{V_b}{V_a}
    \end{equation*}
    \item Molar heat capacities at constant volume ($C_V$) vs. constant pressure ($C_P$):
    \begin{align*}
        C_V &= \frac{1}{n}\dv{Q_V}{T}&
        C_P &= \frac{1}{n}\dv{Q_P}{T}
    \end{align*}
    \begin{itemize}
        \item These quantities are related by considering that
        \begin{align*}
            \dd{E_\text{int}} &= \dd{Q_V}
                = nC_V\dd{T}&
            \dd{E_\text{int}} &&= \dd{Q_P}-p\dd{V}
                = nC_P\dd{T}-p\dd{V}
        \end{align*}
        and hence, algebraically,
        \begin{align*}
            C_P-C_V &= R&
            C_V &= \frac{1}{n}\dv{T}(\frac{3}{2}nRT)
                = \frac{3}{2}R
        \end{align*}
        \item Don't forget the equipartition of energy theorem (every DOF [translational, rotational, vibrational] contributes $kT/2$ of energy).
    \end{itemize}
    \item Coefficient of linear expansion.
    \begin{equation*}
        \alpha = \frac{1}{L}\dv{L}{T}
    \end{equation*}
    \item Adiabatic process: $\dd{Q}=0$.
    \item Ratio of specific heats: $\gamma=C_P/C_V$.
    \begin{itemize}
        \item Varies for different kinds of gases (e.g., monoatomic, diatomic with rotation, diatomic with rotation and vibration).
    \end{itemize}
    \item For an adiabatic process,
    \begin{equation*}
        TV^{\lambda-1} = \text{constant}
    \end{equation*}
    \begin{itemize}
        \item Derive this by noting that $\dd{Q}=0$ and using transitivity between $\dd{E_\text{int}}=-P\dd{V}=nC_V\dd{T}$ where $P=nRT/V$.
    \end{itemize}
    \item $W_\text{by engine}=Q_\text{in}-Q_\text{out}$ per cycle.
    \item Huygens principle: Each part of a wavefront acts as a point source of spherical wavelets.
    \begin{itemize}
        \item Theoretical basis for angle of incidence being equal to angle of reflection.
    \end{itemize}
    \item Index of refraction: $n=c/v$.
    \begin{itemize}
        \item $n$ is a function of frequency, hence why blue light gets diffracted more than red light in a prism.
    \end{itemize}
    \item IOR plus Huygens principle allows us to derive the bending of incoming light waves.
    \item Snell's law:
    \begin{equation*}
        n_1\sin\theta_1 = n_2\sin\theta_2
    \end{equation*}
    \begin{itemize}
        \item Implies a critical angle such that for $\theta_\text{inc}>\theta_\text{crit}$, you only get reflection; no refraction.
    \end{itemize}
    \item Fermat's principle: Light follows the path that takes the least time.
    \item Law of Malus: The following formula, where $I_2$ is the intensity of light that gets past the polaroid filter and $I_1$ is the initial intensity.
    \begin{equation*}
        I_2 = I_1\cos^2\phi
    \end{equation*}
    \begin{itemize}
        \item Averaging $\cos^2\phi$ again, we get
        \begin{equation*}
            I_\text{trans} = \frac{1}{2}I_\text{unpol}
        \end{equation*}
    \end{itemize}
    \item For a spherical mirror, the focal length $f=R/2$.
    \begin{itemize}
        \item Ray tracing to find where the image of an object is.
    \end{itemize}
    \item Mirror equation.
    \begin{equation*}
        \frac{1}{s}+\frac{1}{s'} = \frac{1}{f}
    \end{equation*}
    \item Lateral magnification.
    \begin{equation*}
        m = -\frac{-s'}{s}
    \end{equation*}
    \item Plano-convex lens.
    \begin{equation*}
        \frac{1}{f} = (n-1)\cdot\frac{1}{R}
    \end{equation*}
    \item Lens maker's equation.
    \begin{equation*}
        \frac{1}{f} = (n-1)\cdot\left( \frac{1}{R_1}+\frac{1}{R_2} \right)
    \end{equation*}
    \item Double slit experiment $\Delta r$ derivation.
    \begin{equation*}
        E(r,t) = 2A\cos(kr_0-\omega t)\cos(\frac{k}{2}\cdot d\sin\theta)
    \end{equation*}
    \item Diffraction: If $\alpha=(\pi a/\lambda)\sin\theta$, then
    \begin{equation*}
        I = I_\text{max}\frac{\sin^2\alpha}{\alpha^2}
    \end{equation*}
    \item Rayleigh criterion: The condition for distinguishing sources of light, given by
    \begin{equation*}
        \theta_\text{sep} \geq \frac{1.22\lambda}{a}
    \end{equation*}
\end{itemize}




\end{document}