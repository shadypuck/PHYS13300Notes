\documentclass[../notes.tex]{subfiles}

\pagestyle{main}
\renewcommand{\chaptermark}[1]{\markboth{\chaptername\ \thechapter\ (#1)}{}}
\setcounter{chapter}{14}

\begin{document}




\chapter{Mechanical Waves}
\section{Course Information}
\begin{itemize}
    \item \marginnote{8/5:}HW 1 will be posted after class. Due Monday at 10 AM.
    \item 2 labs, 2 days each.
    \begin{itemize}
        \item Department policy is that you have to do all the labs to pass the class.
    \end{itemize}
    \item First meeting with lab TA will be on Monday at 2:30, 3:30, or 4:30.
    \begin{itemize}
        \item Email Dr. Gazes for later timeslot.
    \end{itemize}
    \item HW accounts for 85\% of your grade because it helps with the tests.
    \item Quiz assignment that you print out, write on, and then scan and upload.
    \item Office hours (Gazes): 5:30-7:00. TA office hours to be posted soon.
    \item Wants us to learn the material, not compete with each other.
    \begin{itemize}
        \item Expects collaboration on the homework, but wants us to write up our own answers.
    \end{itemize}
\end{itemize}



\section{Wave Basics}
\begin{itemize}
    \item \textbf{Wave}: A disturbance that propagates (carrying energy).
    \item \textbf{Mechanical} (wave): A wave in a medium that has an equilibrium.
    \begin{itemize}
        \item Air, for instance, is in equilibrium when its pressure/density is everywhere equal. But you can create a disturbance by making a high-pressure region somewhere in space. This disturbance then propagates.
        \item When a slinky compression wave is created, what's propagating isn't the slinky --- no coil can move past another. What's moving is the \emph{high-density region}.
    \end{itemize}
    \item \textbf{Compression}: The high-density region of a wave.
    \item \textbf{Rarefaction}: The low-density region of a wave.
    \item \textbf{Longitudinal} (wave): A wave where the disturbance is parallel to the propagation of the wave.
    \begin{itemize}
        \item Example: Compression wave in a slinky; air (sound).
    \end{itemize}
    \item \textbf{Transverse} (wave): A wave where the disturbance is perpendicular to the propagation of the wave.
    \begin{itemize}
        \item Example: A string tied to the wall where you shake one end; waves at the beach (the water is going up and down but the wave is moving toward the beach).
    \end{itemize}
    \item A charge $q$ creates an electric field. If $q$ moves at a constant velocity $v$, it will create a magnetic field. If you make the charge accelerate with acceleration $a$, it will produce an \textbf{electromagnetic wave}.
    \item \textbf{Electromagnetic} (wave): A wave that does not require a medium to move in.
    \begin{itemize}
        \item A medium is physical; made up of matter. The electric and magnetic fields in which an electromagnetic wave moves are not media --- they can contain energy, but not in the same way a physical medium can.
    \end{itemize}
    \item \textbf{Wavefunction}: A mathematical function that represents the behavior of a wave.
    \begin{itemize}
        \item $y(x,t)$ represents a one-dimensional wave, $x$ being position and $t$ being time.
        \item $y$ represents the magnitude of the disturbance.
        \begin{itemize}
            \item Example: The density of slinky links in a longitudinal wave; the displacement of a transverse wave from the $x$-axis, taken to be equilibrium.
        \end{itemize}
    \end{itemize}
    \item \textbf{Wave speed}: The velocity with which the wave propagates. \emph{Denoted by} $\bm{v}$.
    \begin{itemize}
        \item NOT, for example, the speed with which the string moves up and down in a transverse wave.
    \end{itemize}
    \item If we let the $xy$-axes be the standard ones, we can also define $x'y'$-axes that move with the wave with velocity $v$.
    \begin{figure}[h!]
        \centering
        \begin{tikzpicture}
            \footnotesize
            \draw [stealth-stealth] (0,2) node[above]{$y$} -- (0,0) -- (5,0) node[right]{$x$};
            \draw [stealth-stealth,xshift=5mm,yshift=-6mm] (0,2) node[above]{$y'$} -- (0,0) -- (5,0) node[right]{$x'$};
            \draw [line width=0.3pt,dashed] (1.5,1.5) -- (1.5,-1.1);
            \node [below right,text height=1.5ex,text depth=0.25ex] at (1.5,0) {$x$};
            \node [below right,text height=1.5ex,text depth=0.25ex] at (1.5,-0.6) {$x'$};
            \draw [very thin] (0,-0.75) -- ++(0,-0.2) ++(0,0.1) -- node[below]{$vt$} ++(0.5,0) ++(0,-0.1) -- ++(0,0.2);
    
            \draw [orx,very thick] (0,0)
                -- (1,0)
                to[out=0,in=180] (1.5,0.8)
                to[out=0,in=180] (2,0)
                -- (4.92,0)
            ;
        \end{tikzpicture}
        \caption{Axes that eliminate the effect of time.}
        \label{fig:axesXprimeYprime}
    \end{figure}
    \begin{itemize}
        \item In the $x'y'$-axes, the wave isn't moving.
        \item From Figure \ref{fig:axesXprimeYprime}, we can see that $x=x'+vt$.
        \item Additionally, we can express (shape of) the wave as $y'=f(x')$.
        \item Thus, $y=f(x-vt)$ represents a wave propagating in the $+x$ direction.
        \item Similarly, $y=f(x+vt)$ represents a wave propagating in the $-x$ direction.
    \end{itemize}
    \item When two waves collide (or we otherwise have to deal with more than one wave in the same medium), we apply the \textbf{superposition principle}.
    \item \textbf{Superposition principle}: If $y_1,y_2,\dots$ are individual wavefunctions, the total disturbance $y$ is given by $y(x,t)=y_1(x,t)+y_2(x,t)+\cdots$.
    \item \textbf{Constructive interference}: When two waves in the same medium add to produce a bigger wave.
    \item \textbf{Destructive interference}: When two waves in the same medium cancel parts of each other out.
    \begin{itemize}
        \item Difference between a medium at equilibrium and a medium with two waves destructively interfering (at the instant the waves collide, the medium looks as if it's at equilibrium):
        \begin{itemize}
            \item The energy of the wave is contained in the kinetic energy of the individual particles of the medium moving up and down.
            \item As such, even when we don't see a visible wave, those particles still have a velocity vector that is containing the energy. It's like the \emph{position} gets back to equilibrium for a moment, but the \emph{velocity}, where the kinetic energy is contained, is most definitely not at equilibrium.
        \end{itemize}
    \end{itemize}
    \item In PHYS 13100, we used $F=ma$ to analyze a block of mass $m$ oscillating on a spring, solving
    \begin{align*}
        F &= ma\\
        -kx &= m\cdot\dv[2]{x}{t}\\
        \dv[2]{x}{t}+\frac{k}{m}\cdot x &= 0
    \end{align*}
    to describe its dynamics.
    \item Creating an analogy to $F=ma$ for wave motion (deriving the wave equation).
    \begin{figure}[h!]
        \centering
        \begin{tikzpicture}
            \footnotesize
            \begin{scope}
                \draw [stealth-stealth] (0,2) node[above]{$y$} -- (0,0) -- (5,0) node[right]{$x$};
                \draw [orx,very thick] (0,0)
                    -- (1,0)
                    to[out=0,in=180] (1.5,0.8)
                    to[out=0,in=180] (2,0)
                    -- (4.92,0)
                ;
                \node at (0.5,0.6) {\normalsize$t$};
            \end{scope}
    
            \begin{scope}[xshift=7cm]
                \draw (0,0) rectangle (4,3);
    
                \fill [orz]
                    ([xshift=1.1cm,yshift=1.3cm]120:2.3mm) to[out=30,in=-170] ([xshift=2.7cm,yshift=1.9cm]100:2.3mm) --
                    ([xshift=2.7cm,yshift=1.9cm]-80:2.3mm) to[out=-170,in=30] ([xshift=1.1cm,yshift=1.3cm]-60:2.3mm) -- cycle
                ;
                \draw [orx,thick]
                    ([xshift=1.1cm,yshift=1.3cm]120:2.3mm) to[out=30,in=-170] ([xshift=2.7cm,yshift=1.9cm]100:2.3mm)
                    ([xshift=2.7cm,yshift=1.9cm]-80:2.3mm) to[out=-170,in=30] ([xshift=1.1cm,yshift=1.3cm]-60:2.3mm)
                ;
                \filldraw [draw=orx,fill=orz,thick,rotate around={30:(1.1,1.3)}] (1.1,1.3) ellipse (1.4mm and 2.3mm);
                \filldraw [draw=orx,fill=orz,thick,rotate around={10:(2.7,1.9)}] ([xshift=2.7cm,yshift=1.9cm]100:2.3mm) arc[start angle=90,end angle=-90,x radius=1.4mm,y radius=2.3mm];
                \draw [orx,thick,rotate around={10:(2.7,1.9)},densely dashed] ([xshift=2.7cm,yshift=1.9cm]100:2.3mm) arc[start angle=90,end angle=270,x radius=1.4mm,y radius=2.3mm];
    
                \draw [blx,semithick,-latex] (1.1,1.3) -- ++(-150:1) node[below,black]{$F$};
                \draw [blx,semithick,-latex] (2.7,1.9) -- ++(10:1) node[above,black]{$F$};
                \draw [blx,semithick,-latex] (2.2,2.1) -- ++(0,0.6) node[right,black]{$a_y$};
    
                \draw [densely dashed,line width=0.3pt] (0.2,1.3) -- (1.1,1.3) node[circle,fill=pix,inner sep=1.2pt]{} -- (1.1,0.4) node[below,text height=1.5ex,text depth=0.25ex]{$x$};
                \draw [densely dashed,line width=0.3pt] (3.7,1.9) -- (2.7,1.9) node[circle,fill=pix,inner sep=1.2pt]{} -- (2.7,0.4) node[below,text height=1.5ex,text depth=0.25ex]{$x+\Delta x$};
    
                \node at (0.5,1.6) {$\alpha_1$}
                    edge [->,out=-170,in=-165,in looseness=1.8] (0.3,1.1)
                ;
                \node at (3.5,1.6) {$\alpha_2$}
                    edge [->,out=10,in=5,in looseness=2] (3.7,2)
                ;
                \node at (1.85,1.65) {$m$};
                \draw [white,double=black,double distance=0.2pt,ultra thick] ([xshift=1.1cm,yshift=1.3cm]-60:0.4)
                    -- ++(-60:0.2)
                    ++(120:0.1)
                    to[out=30,in=-170] node[below,black]{$s$} ([xshift=2.7cm,yshift=1.9cm]-80:0.5)
                    ++(-80:0.1)
                    -- ++(100:0.2)
                ;
            \end{scope}
            
            \draw [white,double=black,double distance=0.4pt,semithick] (1.27,0.67) rectangle (1.53,0.87);
            \draw [->] (1.4,0.87) -- (1.4,1.5) -- (7,1.5);
        \end{tikzpicture}
        \caption{Deriving the wave equation.}
        \label{fig:waveEquationDerivation}
    \end{figure}
    \begin{itemize}
        \item $F$ is a tension force.
        \item We know for the sliver of the string in Figure \ref{fig:waveEquationDerivation}, $F_y=ma_y$.
        \item From our FBD, we have that $F_y=F\sin\alpha_2-F\sin\alpha_1$.
        \item Since the string segment is short, assume $\alpha_1=\alpha_2$. Let's also ignore gravity since $F>>F_g$: it doesn't matter in what position you play an instrument, relative to the Earth's surface, does it?
        \item For small values of $\alpha$ (we assume our string is taut), $\sin\alpha\approx\tan\alpha=\pdv{y}{x}$.
        \item Thus, $F_y=F(\pdv*{y}{x}|_{x+\Delta x}-\pdv*{y}{x}|_{x})$.
        \item Additionally, $m=Ms$, where $M$ is the linear mass density and $s$ is the arc length of the string segment. Furthermore, since $\alpha$'s are small in taut strings, $\Delta s\approx\Delta x$, so $m\approx M\Delta x$.
        \item Lastly, observe that $a_y=\pdv*[2]{y}{t}$.
        \item Therefore, $F=ma$ becomes
        \begin{align*}
            F\left( \eval{\pdv{y}{x}}_{x+\Delta x}-\eval{\pdv{y}{x}}_x \right) &= M\Delta x\cdot\pdv[2]{y}{t}\\
            \frac{\eval{\pdv{y}{x}}_{x+\Delta x}-\eval{\pdv{y}{x}}_x}{\Delta x} &= \frac{M}{F}\cdot\pdv[2]{y}{t}
        \end{align*}
        from which we can take limits as follows:
        \begin{align*}
            \lim_{\Delta x\to 0}\frac{\eval{\pdv{y}{x}}_{x+\Delta x}-\eval{\pdv{y}{x}}_x}{\Delta x} &= \lim_{\Delta x\to 0}\frac{M}{F}\cdot\pdv[2]{y}{t}\\
            \Aboxed{\pdv[2]{y}{x} &= \frac{M}{F}\cdot\pdv[2]{y}{t}}
        \end{align*}
    \end{itemize}
    \item \textbf{Wave equation}: The final result above.
    \begin{itemize}
        \item Holds for a 1D wave on a string.
    \end{itemize}
    \item Tie a piece of string to a wall and shake the free end like a harmonic oscillator. This creates a \textbf{harmonic} wave that propagates towards the wall.
    \item \textbf{Harmonic} (wave): A wave produced by a disturbance changing like a harmonic oscillator.
    \begin{itemize}
        \item The wavefunction for a harmonic wave is sinusoidal, propagates like a wave (i.e., like $f(x-vt)$), and needs to have a constant $k$ to make the dimensional argument of sine dimensionless: $y(x,t)=A\sin(k[x-vt])$.
    \end{itemize}
    \item \textbf{Amplitude}: The constant $A$ in the wavefunction of a harmonic wave.
    \item \textbf{Wavenumber}: The constant $k$ in the wavefunction of a harmonic wave. \emph{Units are} $\si[per-mode=power]{\per\meter}$.
    \item \textbf{Wavelength}: The \emph{distance} over which wave motion repeats \emph{for a fixed time $t$}. \emph{Denoted by} $\bm{\lambda}$.
    \begin{itemize}
        \item Mathematically, the existence of the wavelength implies that $y(x,t)=y(x+\lambda,t)$.
        \item But for a harmonic wave, this implies that $A\sin(k[x-vt])=A\sin(k[(x+\lambda)-vt])$, meaning that $k\lambda=2\pi$.
        \item Thus, we know that the wave number $k=\frac{2\pi}{\lambda}$.
    \end{itemize}
    \item \textbf{Period}: The \emph{time} over which wave motion repeats \emph{for a fixed point $x$}. \emph{Denoted by} $\bm{T}$.
    \begin{itemize}
        \item Similarly, $y(x,t)=y(x,t+T)$.
        \item For a harmonic wave, $A\sin(k[x-vt])=A\sin(k[x-v(t+T)])$, meaning that $kvT=2\pi$.
        \item Thus, we know that the wave speed $v=\frac{2\pi}{k}\cdot\frac{1}{T}=\lambda f$, where $f$ is the frequency of the wave, for simple harmonic motion.
        \item Alternately, if we let $\omega=2\pi f$ be the angular frequency, then $v=\frac{\omega}{k}$.
    \end{itemize}
    \item It follows that for a harmonic wave,
    \begin{align*}
        y(x,t) &= A\sin\left[ 2\pi\left( \frac{x}{\lambda}-\frac{t}{T} \right) \right]\\
        &= A\sin[kx-\omega t]\footnotemark
    \end{align*}
    \footnotetext{Dr. Gazes prefers this form, but both are correct and can be used.}
    \item To account for cosine and other waves that "start" at different parts, we include a \textbf{phase constant} $\phi$:
    \begin{equation*}
        y(x,t) = A\sin[kx-\omega t+\phi]
    \end{equation*}
    \item To check that the above is in fact a wave, we must feed it into the wave equation:
    \begin{align*}
        \pdv[2]{x}(A\sin[kx-\omega t+\phi]) &= \frac{M}{F}\cdot\pdv[2]{t}(A\sin[kx-\omega t+\phi])\\
        -Ak^2\sin[kx-\omega t+\phi] &= \frac{-A\omega^2M}{F}\sin[kx-\omega t+\phi]\\
        k^2 &= \frac{M\omega^2}{F}\\
        \frac{\omega}{k} &= \sqrt{\frac{F}{M}}
    \end{align*}
    \begin{itemize}
        \item It follows since $v=\frac{\omega}{k}$ that $v=\sqrt{F/M}$.
    \end{itemize}
    \item We originally found this speed/force/mass relationship to be true for a harmonic wave, but this shows that it is true for \emph{any} wave.
    \item \textbf{General 1D wave equation}: Making the modification from above, the following equation.
    \begin{equation*}
        \pdv[2]{y}{x} = \frac{1}{v^2}\cdot\pdv[2]{y}{t}
    \end{equation*}
\end{itemize}



\section{Office Hours (Gazes)}
\begin{itemize}
    \item How does proving that $v=\sqrt{F/M}$ with a harmonic wavefunction prove that this relation holds for \emph{all} waves?
    \begin{itemize}
        \item Applies to any wave in a string. If you have a shape that doesn't look like a harmonic wave, you can construct it out of harmonic waves (Fourier math). The superposition principle allows us to add these waves.
    \end{itemize}
    \item Importance of reading the textbook?
    \begin{itemize}
        \item To be used as we wish.
        \item We \emph{could} read it instead of coming to lecture.
        \item Think of it as something to consult as needed; i.e., for clarification.
        \item Some people read it before class.
        \item He will talk about some things in class that aren't in the textbook, and vice versa. If the textbook talks about it and he doesn't, you aren't responsible for knowing it.
    \end{itemize}
\end{itemize}



\section{Wave Dynamics}
\begin{itemize}
    \item \marginnote{8/6:}TA office hours on Wednesday and Sunday; 2 timeslots on both days.
    \item Dr. Gazes will post lab sections this afternoon.
    \item \textbf{Transverse velocity}: The speed at which a fixed point in the medium through which a transverse wave travels moves up and down. \emph{Given by}
    \begin{align*}
        v_t = \pdv{y}{t}
    \end{align*}
    \begin{itemize}
        \item For a harmonic wave, $v_t=\omega A\cos(kx-\omega t+\phi)$.
    \end{itemize}
    \item \textbf{Transverse acceleration}: The acceleration of a fixed point in the medium through which a transverse wave travels. \emph{Given by}
    \begin{align*}
        a_t = \pdv{v_t}{t}
    \end{align*}
    \begin{itemize}
        \item For a harmonic wave, $a_t=-\omega^2A\sin(kx-\omega t+\phi)$.
    \end{itemize}
    \item When a point achieves its maximum positive displacement $y=+A$, it has $v_t=0$ and $a_t=-\omega^2A$.
    \begin{itemize}
        \item Similarly, at $y=-A$, it still has $v_t=0$, but it also has $a_t=\omega^2A$.
        \item When a point has zero displacement ($y=0$), it has $v_t=\pm\omega A$ and $a_t=0$.
    \end{itemize}
    \item $y$ and $a_t$ are $\ang{180}$ out of phase with each other.
    \item $y$ and $v_t$ are $\ang{90}$ out of phase with each other.
    \item \textbf{Power}: The rate at which a wave carries energy. \emph{Given by}
    \begin{equation*}
        P = \dv{W}{t}
        = \dv{E}{t}
        = \vec{F}\cdot\vec{v}
    \end{equation*}
    \item Wave energy:
    \begin{itemize}
        \item Kinetic: $K=\frac{1}{2}mv_t^2$ for each little sliver of the string.
        \begin{itemize}
            \item Thus, since $v=\omega A=2\pi fA$, we have that $K\propto \omega^2,f^2,A^2$.
            \item Additionally, since $P=\vec{F}\cdot\vec{v}=F\cdot(2\pi fA)$, we have that $P\propto v,f^2,A^2$.
        \end{itemize}
        \item Places where the string crosses the equilibrium axis have maximum stretching, i.e., potential energy.
    \end{itemize}
    \item When you shake a string attached to a wall, the power $P_\text{hand}$ exerted by your hand is given by
    \begin{figure}[h!]
        \centering
        \begin{tikzpicture}
            \footnotesize
            \shade [left color=gax,right color=white] (2*pi,-1) rectangle ++(0.2,2);
            \draw [thick] (2*pi,-1) -- ++(0,2);
    
            \draw [ory,very thick,densely dashed,scale=0.5] (0,0) -- (4*pi,0);
            \draw [orx,very thick,scale=0.5] (0,0) plot[domain=0:4*pi,smooth] (\x,{sin(\x r)});
    
            \draw [very thin,densely dashed] (-1,0) coordinate (A) -- (0,0) coordinate (B);
            \draw [grx,semithick,latex-latex] (0,-0.6) -- (0,0.6) node[right,black]{$v_t$};
            \draw [blx,thick,-latex] (0,0) -- ++(-135:1) coordinate (C) node[below,black]{$F_\text{hand}$}
                pic [black,fill=blz,angle eccentricity=0.7,pic text={$\alpha$},pic text options={yshift=0.6pt}] {angle=A--B--C}
            ;
            \fill [pix] (0,0) circle (2pt);

            \fill (2*pi,2pt) arc[start angle=90,end angle=270,radius=2pt];
    
            \draw [grx,ultra thick,-latex] (4.7,0.4) -- ++(0.7,0) node[above,black]{$v$};
            \draw [stealth-stealth] (-0.9,1.2) node[above]{$y$} -- ++(0,-0.4) node[above right=-1pt]{\scriptsize$+$} -- ++(0.4,0) node[right]{$x$};
        \end{tikzpicture}
        \caption{Power of a wave.}
        \label{fig:wavePower}
    \end{figure}
    \begin{align*}
        P_\text{hand} &= \vec{F}_\text{hand}\cdot\vec{v}\\
        &= F_{\text{hand},y}v_t\\
        &= (F\cdot -\sin\alpha)\cdot\left( \pdv{y}{t} \right)\\
        &\approx (-F\tan\alpha)\cdot\left( \pdv{y}{t} \right)\\
        &= \left( -F\cdot\pdv{y}{x} \right)\cdot\left( \pdv{y}{t} \right)\\
        &= (-F\cdot kA\cos(kx-\omega t+\phi))\cdot(-\omega A\cos(kx-\omega t+\phi))\\
        &= Fk\omega A^2\cos^2(kx-\omega t+\phi)\\
        &= Mv^2k\omega A^2\cos^2(kx-\omega t+\phi)\\
        &= Mv\omega^2A^2\cos^2(kx-\omega t+\phi)
    \end{align*}
    \begin{itemize}
        \item Thus, since the average value of $\cos^2(x)=\frac{1}{2}$, the average power $\bar{P}$ of a wave on a string is given by
        \begin{equation*}
            \bar{P} = \frac{1}{2}Mv\omega^2A^2
        \end{equation*}
        \begin{itemize}
            \item Increasing the amplitude of a wave increases the power of the wave without changing the frequency or wave speed.
            \item This is what radio stations do to boost the power of their broadcast (since they can't change the speed of light and changing the frequency would change their channel).
        \end{itemize}
    \end{itemize}
    \item \textbf{Compound string}: Two pieces of string (of differing composition) attached together.
    \item When an incident wave encounters a change of medium, it both transmits \emph{and} reflects in parts.
    \begin{itemize}
        \item The "knot" moving up and down is the source of the transmitted and reflected waves.
    \end{itemize}
    \item Compound string analysis:
    \begin{figure}[H]
        \centering
        \begin{tikzpicture}
            \footnotesize
            \draw [ory,thick] (-3.5,0) to[out=0,in=180] ++(0.5,0.8) node[above,black]{incident} to[out=0,in=180] ++(0.5,-0.8);
            \draw [ory,thick] (-1.5,0) to[out=0,in=180] ++(0.5,0.8) node[above,black]{reflected} to[out=0,in=180] ++(0.5,-0.8);
            \draw [ory,line width=2pt] (0.5,0) to[out=0,in=180] ++(0.5,0.8) node[above,black]{transmitted} to[out=0,in=180] ++(0.5,-0.8);
    
            \draw [orx,thick] (-4,0) node[left,label={left:${\color{black}-\infty}$}]{\large$\cdots$} -- node[below=4pt,black]{$\mu_1$} (0,0);
            \draw [orx,line width=2pt] (0,0) node[circle,fill=pix,inner sep=1.5pt,label={below:${\color{black}x=0}$}]{} -- node[below=4pt,black]{$\mu_2$} (4,0) node[right,label={[xshift=-1mm]right:${\color{black}+\infty}$}]{\large$\cdots$};
    
            \draw [grx,ultra thick,-latex] (-2.6,0.5) -- ++(0.5,0);
            \draw [rex,ultra thick,-latex] (-1.4,0.5) -- ++(-0.4,0);
            \draw [grx,ultra thick,-latex] (1.4,0.5) -- ++(0.4,0);

            \draw [stealth-stealth] (-4.5,1.2) node[above]{$y$} -- ++(0,-0.4) node[above right=-1pt]{\scriptsize$+$} -- ++(0.4,0) node[right]{$x$};
        \end{tikzpicture}
        \caption{Compound string waves.}
        \label{fig:compoundString}
    \end{figure}
    \begin{itemize}
        \item General wave equations for the incident wave ($y_i$), the transmitted wave ($y_t$), and the reflected wave ($y_r$):
        \begin{gather*}
            y_i(x,t) = A_i\cos(k_1x\pm\omega_1t)\\
            y_t(x,t) = A_t\cos(k_2x\pm\omega_2t)\\
            y_r(x,t) = A_r\cos(k_3x\pm\omega_3t)
        \end{gather*}
        \item According to the coordinate system in Figure \ref{fig:compoundString}, we choose $-,-,+$ from top to bottom for our wave equations.
        \item $\omega_1=\omega_2=\omega_3$ because the frequency of the incident wave will be the frequency with which the knot moves.
        \item $k=\frac{\omega}{v}=\omega\sqrt{M/F}$ varies because while $\omega$ and the tension force are the same (the latter because otherwise the knot would be accelerating), the linear mass density varies.
        \begin{itemize}
            \item However, since the incident and reflected waves move in the same medium, $k_1=k_3$.
        \end{itemize}
        \item Boundary conditions:
        \begin{enumerate}
            \item String doesn't break, so $y$ is continuous at $x=0$.
            \item String has no kinks (because then you would have a point of zero mass with an unbalanced force on it, leading to an infinite acceleration, which is impossible), so $\pdv*{y}{x}$ is continuous at $x=0$.
        \end{enumerate}
        \item Thus, since
        \begin{equation*}
            y =
            \begin{cases}
                y_i+y_r & x<0\\
                y_t & x>0
            \end{cases}
        \end{equation*}
        boundary condition 1 implies that $y_i(0,t)+y_r(0,t)=y_t(0,t)$ for all $t$. Consequently,
        \begin{align*}
            A_i\cos(k_1(0)-\omega t)+A_r\cos(k_1(0)+\omega t) &= A_t\cos(k_2(0)-\omega t)\\
            A_i\cos(-\omega t)+A_r\cos(-\omega t) &= A_t\cos(-\omega t)\\
            A_i+A_r &= A_t
        \end{align*}
        \item Additionally, boundary condition 2 implies that $\eval{\pdv*{y_i}{x}}_{x=0}+\eval{\pdv*{y_r}{x}}_{x=0}=\eval{\pdv*{y_t}{x}}_{x=0}$ for all $t$. Consequently,
        \begin{align*}
            \eval{\pdv{x}(A_i\cos(k_1x-\omega t))}_{x=0}+\eval{\pdv{x}(A_r\cos(k_1x+\omega t))}_{x=0} &= \eval{\pdv{x}(A_t\cos(k_2x-\omega t))}_{x=0}\\
            \eval{-A_ik_1\sin(k_1x-\omega t)}_{x=0}+\eval{-A_rk_1\sin(k_1x+\omega t)}_{x=0} &= \eval{-A_tk_2\sin(k_2x-\omega t)}_{x=0}\\
            -A_ik_1\sin(-\omega t)-A_rk_1\sin(\omega t) &= -A_tk_2\sin(-\omega t)\\
            -A_ik_1\sin(-\omega t)+A_rk_1\sin(-\omega t) &= -A_tk_2\sin(-\omega t)\\
            -A_ik_1+A_rk_1 &= -A_tk_2\\
            k_1(A_i-A_r) &= k_2A_t
        \end{align*}
        \item It follows by solving like a system of equations that
        \begin{align*}
            \frac{A_r}{A_i} &= \frac{k_1-k_2}{k_1+k_2}&
            \frac{A_t}{A_i} &= \frac{2k_1}{k_1+k_2}
        \end{align*}
        \item This combined with the fact that $k_1\propto\sqrt{\mu_1}$ and $k_2\propto\sqrt{\mu_2}$ implies that
        \begin{align*}
            \frac{A_r}{A_i} &= \frac{\sqrt{\mu_1}-\sqrt{\mu_2}}{\sqrt{\mu_1}+\sqrt{\mu_2}}&
            \frac{A_t}{A_i} &= \frac{2\sqrt{\mu_1}}{\sqrt{\mu_1}+\sqrt{\mu_2}}
        \end{align*}
    \end{itemize}
    \item Let's run a few checks on some special cases.
    \begin{itemize}
        \item Let $\mu_1=\mu_2$, i.e., the compound string is a uniform string. Then $A_r/A_i=0$ and $A_t/A_i=1$, as we would expect.
        \item Let $\mu_1<<\mu_2$, i.e., one string is tied to an immovable wall. Then $A_r/A_i\to -1$ and $A_t/A_i\to 0$, as we would expect by Newton's third law.
        \item Let $\mu_1>>\mu_2$. Then $A_r/A_i\to 1$ and $A_t/A_i\to 2$.
    \end{itemize}
    \item Suppose you have a string tied between two walls.
    \begin{figure}[h!]
        \centering
        \begin{tikzpicture}[
            every node/.append style={black}
        ]
            \footnotesize
            \draw [yshift=-1.5cm,-stealth]
                (0,0.1) -- ++(0,-0.2) node[below]{$0$}
                (6,0.1) -- ++(0,-0.2) node[below]{$L$}
                (-0.5,0) -- (6.5,0) node[right]{$x$}
            ;
            
            \draw [ory,very thick] (0.5,0) to[out=0,in=180] ++(0.5,-0.8) node[below]{$y_-$} to[out=0,in=180] ++(0.5,0.8);
            \draw [ory,very thick] (4.5,0) to[out=0,in=180] ++(0.5,0.8) node[above]{$y_+$} to[out=0,in=180] ++(0.5,-0.8);
            \draw [rex,ultra thick,-latex] (4.6,0.5) -- ++(-0.7,0);
            \draw [grx,ultra thick,-latex] (1.4,-0.5) -- ++(0.7,0);

            \draw [orx,very thick] (0,0) -- (6,0);

            \shade [left color=white,right color=gax] (0,-1) rectangle ++(-0.2,2);
            \shade [left color=gax,right color=white] (6,-1) rectangle ++(0.2,2);
            \draw [thick]
                (0,1) -- (0,-1)
                (6,1) -- (6,-1)
            ;
            \fill
                (0,2pt) arc[start angle=90,end angle=-90,radius=2pt]
                (6,2pt) arc[start angle=90,end angle=270,radius=2pt]
            ;
        \end{tikzpicture}
        \caption{A string tied between two walls.}
        \label{fig:stringWalls}
    \end{figure}
    \begin{itemize}
        \item If you send a wave $y_+$ in the $-x$-direction, it will be reflected and inverted in its entirety at the left wall into the wave $y_-$.
        \item This yields a total wavefunction
        \begin{align*}
            y &= y_++y_-\\
            &= A\cos(kx+\omega t)-A\cos(kx-\omega t)\\
            &= A[\cos(kx+\omega t)-\cos(kx-\omega t)]\\
            &= 2A\sin\left( \frac{(kx+\omega t)+(kx-\omega t)}{2} \right)\sin\left( \frac{(kx+\omega t)-(kx+\omega t)}{2} \right)\\
            &= 2A\sin(kx)\sin(\omega t)
        \end{align*}
        \item Boundary conditions:
        \begin{enumerate}
            \item $y(0,t)=0$ for all $t$.
            \item $y(L,t)=0$ for all $t$.
        \end{enumerate}
        \item From the second boundary condition, we know that we must have $\sin(kL)=0$, i.e., $kL=n\pi$ for some $n\in\mathbb{N}$ (the wavenumber cannot be negative or zero by definition).
        \item Thus, $k_n=\frac{n\pi}{L}$.
        \item It follows since $k=\frac{2\pi}{\lambda}$ that $L=\frac{n}{2}\cdot\lambda$.
    \end{itemize}
    \item More specifically, if $L=\frac{n}{2}\cdot\lambda$ for some $n\in\mathbb{N}$, then we will have a \textbf{standing wave}.
    \item \textbf{Node}: A point in the medium of a standing wave with amplitude zero.
    \item \textbf{Antinode}: A point in the medium of a standing wave with maximum amplitude.
    \item Frequency of standing waves:
    \begin{figure}[H]
        \centering
        \begin{subfigure}[b]{0.49\linewidth}
            \centering
            \begin{tikzpicture}[
                every node/.append style={black}
            ]
                \footnotesize
                \draw [ory,very thick,dashed] (0,0) -- (6,0);
                \draw [orx,very thick]
                    (0,0) sin (3,0.7) cos (6,0)
                    (0,0) sin (3,-0.7) cos (6,0)
                ;
        
                \shade [left color=white,right color=gax] (0,-1) rectangle ++(-0.2,2);
                \shade [left color=gax,right color=white] (6,-1) rectangle ++(0.2,2);
                \draw [thick]
                    (0,1) -- (0,-1)
                    (6,1) -- (6,-1)
                ;
                \fill
                    (0,2pt) arc[start angle=90,end angle=-90,radius=2pt]
                    (6,2pt) arc[start angle=90,end angle=270,radius=2pt]
                ;
            \end{tikzpicture}
            \caption{$n=1$.}
            \label{fig:harmonicFrequenciesa}
        \end{subfigure}
        \begin{subfigure}[b]{0.49\linewidth}
            \centering
            \begin{tikzpicture}[
                every node/.append style={black}
            ]
                \footnotesize
                \draw [ory,very thick,dashed] (0,0) -- (6,0);
                \draw [orx,very thick]
                    (0,0) sin (1.5,0.7) cos (3,0) sin (4.5,-0.7) cos (6,0)
                    (0,0) sin (1.5,-0.7) cos (3,0) sin (4.5,0.7) cos (6,0)
                ;
        
                \shade [left color=white,right color=gax] (0,-1) rectangle ++(-0.2,2);
                \shade [left color=gax,right color=white] (6,-1) rectangle ++(0.2,2);
                \draw [thick]
                    (0,1) -- (0,-1)
                    (6,1) -- (6,-1)
                ;
                \fill
                    (0,2pt) arc[start angle=90,end angle=-90,radius=2pt]
                    (6,2pt) arc[start angle=90,end angle=270,radius=2pt]
                ;
            \end{tikzpicture}
            \caption{$n=2$.}
            \label{fig:harmonicFrequenciesb}
        \end{subfigure}
        \caption{Fundamental harmonic frequencies.}
        \label{fig:harmonicFrequencies}
    \end{figure}
    \vspace{-1em}
    \begin{align*}
        f &= \frac{v}{\lambda}\\
        &= \sqrt{\frac{F}{M}}\cdot\frac{n}{2L}
    \end{align*}
    \begin{itemize}
        \item When $n=1$, we call $f_1=\frac{1}{2L}\sqrt{F/M}$ the \textbf{first fundamental harmonic frequency}.
        \item When $n=2$, we call $f_1=2f_1$ the \textbf{second fundamental harmonic frequency}.
        \item Similarly, $f_n=nf_1$ for the \textbf{$\bm{n^\text{th}}$ fundamental harmonic frequency}.
    \end{itemize}
    \item Different instruments have different \textbf{overtones} (combinations of harmonics).
    \item We have some lab stuff to do before Monday.
    \item If you vibrate a string at a certain frequency, you can build up energy in the wave. Otherwise, you will just have all sorts of dissonant destructive interference. Think about pushing a swing --- you have to push it at the right time to build up a big amplitude.
\end{itemize}




\end{document}